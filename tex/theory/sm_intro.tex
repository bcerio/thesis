
The SM seeks to predict the spectrum of, and
interactions among, the particles that constitute matter. Since these
elementary particles are infinitesimally small, their behavior is
governed by the postulates of quantum mechanics (QM). More
specifically, the SM is built on a theoretical framework known as
quantum field theory (QFT), a relativistic extension of classical
QM. QFT improves on classical QM by allowing particle number
to be violated in a closed system, a phenomenon that is observed in,
for example, an atomic energy level transition whereby a photon is
created or annihiliated. Particles in QFT are described as excited states
of space-time fields that are realized as mathematical
operators. These operators serve to create or annihilate particles. The
dynamics of a field, and its associated
particles, is obtained from a Lagrangian density,
$\mathcal{L}(\phi,\partial_{\mu}\phi)$, the field-theoretic
analogue of the Lagrangian in classical mechanics. In the context of
QFT, if two different fields appear in the same term of $\mathcal{L}$, the two
particles associated with these fields are said to ``couple''. It is
these couplings that lead to the fundamental interactions. 

Elementary particles broadly fall into two categories: half-integer
spin fermions and integer spin bosons. Quarks and leptons, for example, are fermions
with spin-1/2, while the particle that mediates the electromagnetic
(EM) interaction, the photon, is a spin-1 boson. Dirac formulated a
relativistic analogue of the Schrodinger equation for spin-1/2
particles, and the corresponding Lagrangian is

\begin{equation}
\mathcal{L}_{\textrm{fermion}} =
i\bar{\psi}\gamma_{\mu}\partial^{\mu}\psi - m\bar{\psi}\psi.
\label{chap:theory:equation:dirac_lagrangian}
\end{equation}

\noindent
To incorporate the two spin states of the fermion and the
fact that each particle has an antiparticle, the field $\psi$ has four
degrees of freedom. In the above equation, $\bar{\psi} =
\psi^{\dagger}\gamma^0$, where $\psi^{\dagger}$ denotes the
adjoint. The $\gamma^{\mu}$ are the Dirac $\gamma$-matrices,
whose anticommutators form the Dirac algebra:
$\left\{\gamma^{\mu},\gamma^{\nu}\right\} = 2g^{\mu\nu}$. The first term
of equation~\ref{chap:theory:equation:dirac_lagrangian} is the kinetic
energy term, and the second is the mass term that gives a mass $m$ to
the particle associated with the field. 

Particles that are excitations of the fields described by
equation~\ref{chap:theory:equation:dirac_lagrangian} do not interact
with eachother, which is empirically known to be false. Charged
electrons experience an electromagnetic force from the atomic nucleus,
whose constituents are bound together by the strong interaction. Some
of these nuclei emit radiation as a consequence of the weak interaction. In the
SM, these three forces are incorporated as additional terms in
$\mathcal{L}_{\textrm{fermion}}$. The fourth fundamental interaction,
gravity, has not been successfully included in the model. Inter-field
interactions arise when a local gauge symmetry is imposed. These
so-called symmetries are in fact just internal mathematical degrees of
freedom that are not expected to impact the observed behavior of a field. For
example, the local phase of a field, $\alpha(x)$, does not correspond
to a physical quantity. Therefore, the Lagrangian should be
invariant under the transformation $\psi\rightarrow
e^{-ig\alpha(x)}\psi$. The non-interacting Lagrangian
in equation~\ref{chap:theory:equation:dirac_lagrangian} does not fulfill this
requirement. However, if a new field is introduced through a
term of the form $g\bar{\psi}\gamma^{\mu}\psi A_{\mu}$,
$\mathcal{L}_{\textrm{fermion}}$ becomes invariant as long as the new
field transforms as $A_{\mu}\rightarrow{A_{\mu}
+ \partial_{\mu}\alpha(x)}$. The vector field
$A_{\mu}$ is called a gauge field, and it represents the photon. 
Because this new field represents a
physical particle, a kinetic energy term associated with $A_{\mu}$
needs to be introduced. A choice that preserves gauge and Lorentz
invariance is a kinetic energy term of the form $-(1/4)F_{\mu\nu}F^{\mu\nu}$,
where $F_{\mu\nu}$ is the EM field strength tensor. A mass term for $A_{\mu}$ is not included in the
Lagrangian, because it breaks gauge invariance. The photon, therefore,
is considered to be massless. It can be shown
that a Lagrangian of this form for $A_{\mu}$ recovers Maxwell's
equations.

An important property of the gauge transformation
$e^{-ig\alpha(x)}$ is that it forms a Lie group under
multiplication~\cite{}. In general, any element of a Lie group can be
written as $e^{-i\alpha_j(x)X_j}$, where the $X_j$ are the group
generators and the parameters $\alpha_j(x)$ identify each element of the
group. For a gauge theory, the number of generators is equal to the
number of gauge fields that arise when invariance is imposed in the
Lagrangian. Furthermore, the nature of the resulting gauge interactions is defined
by the commutation relations among the generators:
$\left[X_i,X_j\right] = if_{ijk}X_k$, where the $f_{ijk}$ are real
constants called the structure constants.

For the EM interaction, the symmetry group is $U(1)$, an Abelian group which
has a single generator, which, as discussed above, results in the photon. For the weak
interaction, the internal symmetry is weak isospin, which is described by the Lie group
$SU(2)$. With three generators, $SU(2)$ gauge invariance produces three
fields associated with the weak vector bosons $W^+$, $W^-$, and
$Z$. In the theory of the strong interaction, called quantum chromodynamics
(QCD), an $SU(3)$ color symmetry is required, resulting in eight gauge
fields associated with gluons. Apart from producing more gauge fields,
the $SU(2)$ and $SU(3)$ symmetries differ from the $U(1)$ symmetry of
the EM interaction because they have non-zero
structure constants. A consequence of this differing group structure
is that self-interaction terms arise in the Lagrangian. Hence, both
gluons and weak vector bosons can couple to themselves. 

In EM, the inclusion of a mass term for $A_{\mu}$ breaks the $U(1)$
gauge symmetry. This is easily resolved by setting the mass of the
photon to zero. The same problem arises in both strong and weak
Lagrangians. In the case of QCD, the eight gluon masses are set to
zero, and color symmetry is restored. The masses of the three weak
gauge bosons, on the other hand, have been measured to be non-zero. In
fact, these particles are quite massive: $m_W = 80.4 \gev$ and $m_Z =
90.1 \gev$. If the gauge symmetry is broken by introducing mass
terms, infinities are induced in the perturbative
expansion of the path integral which can not be renormalized,
rendering the theory non-predictive. 

In addition to the weak
boson mass terms, the fermion mass terms
in the weak Lagrangian violate the $SU(2)$ symmetry. Since the weak
interaction has been measured to maximally violate the discrete
symmetry known as parity, gauge invariance is only required for
left-handed fields, though both left and right-handed fermion fields
exist in nature and hence, in the Lagrangian. With a mixture of
helicity states in the mass term, each transforming differently under
$SU(2)$, the symmetry is broken. The prediction implied by setting all
fermion masses to zero is at odds with a myriad of experimental
evidence. These theoretical problems with the weak boson and fermion
masses led to the development of spontaneous symmetry breaking. 

An approach for simultaneously introducing both gauge boson and
fermion mass terms
into the Lagrangian, while preserving gauge symmetry and
renormalizability, was developed in the 1960s~\cite{}, and was
subsequently adapted for the unified electroweak theory~\cite{}. A
complex scalar field that transforms as a weak isospin doublet, and is
described by the Lagrangian

\begin{equation}
\mathcal{L}_{\textrm{Higgs}} =
\left(D_{\mu}\phi^{\dagger}\right)\left(D^{\mu}\phi\right)
- \left[ \mu^2 \phi^{\dagger}\phi + \lambda (\phi^{\dagger}\phi)^2\right]
\label{chap:theory:equation:higgs_lagrangian}
\end{equation}

\noindent
is introduced into the Lagrangian describing both EM and weak
interactions. This form is chosen to be invariant under the gauge
symmetry $SU(2) \times U(1)$, and due to the restrictions $\mu^2 < 0$
and $\lambda > 0$, to yield non-zero $\phi$ at the potential energy
minimum. In the QFT context, the $\phi$ value that minimizes the
potential energy, denoted $\phi_0$, corresponds to the physical
absence of particle excitations associated with the field, and is
therefore known as the vacuum expectation value (vev). The manifold on
which the potential is minimized is, from the invariance of
$\mathcal{L}_{\textrm{Higgs}}$, manifestly invariant under
$SU(2) \times U(1)$. The $SU(2)$ symmetry is then broken ``spontaneously''
when a point on the manifold is chosen, or, equivalently, when a gauge
is chosen. This process is spontaneous in the sense that there is not
a dynamical theory for why it occurs; it is merely postulated that it
has occurred. 

By performing a gauge transformation on the resulting isospin doublet
$\phi_0$, the non-zero expectation value can be projected into the
real part of the neutral component of the doublet, leaving the three
remaining real scalar degrees of freedom with a magnitude of zero.
To recover the dynamics, the remaining scalar component is perturbed
about the vacuum expectation value--- $v + H(x)$--- where the field
$H(x)$ is a real scalar field known as the Higgs field. Putting this
expression for $\phi$ into
equation~\ref{chap:theory:equation:higgs_lagrangian}, mass terms for
$W^{\pm}$ and $Z$ are generated in the mixing of the $v$ part of the
field and bilinear gauge boson terms in the covariant derivative
$D_{\mu}$. Additional gauge invariant terms that mix $\phi$
and bilinear fermion terms are added to the Lagrangian, resulting in
mass terms when $\phi$ acquires a vev. This mechanism, therefore, is successful in mitigating the two
deficiencies outlined above. An important bi-product of this mechanism
for generating masses is the prediction of the existence of a spin-0
particle, hereafter referred to as the Higgs boson or the
Higgs. The Higgs boson couples to fermions and gauge bosons with a
strength proportional to their masses, implying that direct production of
such a particle is possible at colliders. Moreover, such couplings
induce higher order corrections in measureable quantities such as the
top quark and $W$ boson masses. Consequently, the existence of the
Higgs boson can be indirectly established by measuring deviations in
these observables. Though the
Lagrangian~\ref{chap:theory:equation:higgs_lagrangian} appears to
introduce two new parameters to the SM, one can be related to existing
SM input parameters, leaving only one parameter that can be expressed
in terms of the mass of the Higgs boson ($m_H$).

The above procedure, known popularly as the Higgs mechanism, restores the gauge
symmetry of the SM Lagrangian, thereby ensuring
renormalizability. The underlying $SU(2)$ symmetry of the weak
interaction is hidden, or spontaneously broken, resulting in a
tangible prediction, the existence of a massive, spin-0 boson whose
coupling to other particles is linear with their masses. Other scalar
degrees of freedom in the field $\phi$ become the longitudinal
polarization of the weak gauge bosons, a necessary property of massive
spin-1 particles. This form of the Higgs mechanism is not unique; the
chosen representation of $SU(2)$, namely that $\phi$ is a complex
scalar doublet, is merely the most parsimonious. Additional degrees of
freedom arise if another representation is chosen or if an additional
$SU(2)$ invariant Higgs fields are introduced. In some cases, these
extended Higgs scenarios, which predict additional Higgs particles,
fix other theoretical limitations of the SM. However, because every
incarnation of the Higgs mechanism has at least one neutral scalar Higgs boson,
the minimalistic form is chosen in the SM. 

In its current form, the SM is a gauge theory obeying $SU(3) \times
SU(2) \times U(1)$ symmetry, resulting in a total of 12 gauge bosons:
8 massless, bi-colored gluons, 3 massive weak vector bosons, and a
single massless photon. Three fermion generations, each with a charged
lepton, a neutrino, two quarks, as well as their corresponding
anti-matter particles, have been observed, totaling 48
fermions. Adding to the list the all-important Higgs boson, 61
particles are predicted to exist by the SM, all of which have been
experimentally observed. 



