As discussed in section XX, gluon fusion is the dominant Higgs production mode at the
LHC, with a cross section that is a factor of \textapprox{10} larger
than VBF. Requiring two jets in the final state, however, suppresses
this background significantly. Furthermore, given that the jets in ggF
events arise from QCD vertices, cuts that isolate the VBF topology
efficiently remove ggF. Nevertheless, because the it is the same Higgs
decay, there is phase space overlap between the signal and this
background, and in the \emme channel, ggF accounts for 10\% of the
total background, assuming a Higgs mass of 125 \GeV.

The ggF$+2$j process is simulated with \POWHEG interfaced with
\PYTHIAns8. The cross section is computed at NNLO in
QCD~\cite{Djouadi:1991tka,Dawson:1990zj,Spira:1995rr,Harlander:2002wh,Anastasiou:2002yz,Ravindran:2003um},
with NLO electroweak corrections~\cite{Aglietti:2004nj,Actis:2008ug}
and soft gluon resummations up to
next-to-next-to-leading-log (NNLL)~\cite{Catani:2003zt}, as recommended by
the LHC Higgs Cross Section Working
Group~\cite{bib:Dittmaier:2011ti,bib:Dittmaier:2012vm,bib:Heinemeyer:2013tqa}. The
branching fraction used in the normalization comes from
\hdecay~\cite{Djouadi:1997yw}, with the associated uncertainties
compiled
in~\cite{bib:Dittmaier:2011ti,bib:Dittmaier:2012vm}. Due
to known deficiencies in the Higgs $p_T$ spectrum in \POWHEG, the
Higgs $p_{\mathrm{T}}$ is re-weighted to the NLO + NNLL prediction
from \HqT~\cite{deFlorian:2011xf}. With these corrections, the cross
section for this decay channel is $\sigma({\hww}) =
\sigma_{\mathrm{total}}(m_H = 125)\mathrm{BR}(\hww) =
19.27~\mathrm{pb}\cdot{0.215} = 4.15$~pb.

Mention using 0,1j to constrain ggF normalization?


