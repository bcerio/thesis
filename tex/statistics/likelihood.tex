
The simplified likelihood function defined in
equation~\ref{chap:statistics:equation:simple_likelihood} does not
account for the fact that (1) some of the components in $b$ are
normalized with control regions, (2) predictions $s$ and $b$ are inherently
uncertain (3) the number of events $N$ is generally measured in
several phase space bins for some observable $x$. 

If the normalization of a background is constrained in a control
region, the normalization factor (NF) can be incorporated as an
additional parameter in the likelihood:

%\begin{equation}
%\mathscr{L}(\mu,\mu_b) = P(N|\mu s + \mu_b b^{\textrm{SR}} +
%b_{\textrm{other}})\cdot P(M|\mu_b b^{\textrm{CR}} + \mu
%s^{\textrm{CR}} + b_{\textrm{other}}^{\textrm{CR}})
%\end{equation}
\begin{equation}
\mathscr{L}(\mu,\mu_b) = P(N\,|\,\mu s + \mu_b b^{\textrm{SR}} +
b_{\textrm{other}}) \cdot P(M\,|\,\mu_b b^{\textrm{CR}})
\label{chap:statistics:equation:likelihood_CR}
\end{equation}

\noindent
The second term represents the auxiliary measurement $M$ in the CR,
with an expectation
%$\mu_b b^{\textrm{CR}}+\mu s^{\textrm{CR}} +
%b_{\textrm{other}}^{\textrm{CR}}$, where the residual signal $\mu
%s^{\textrm{CR}}$ and other background components
%$b_{\textrm{other}}^{\textrm{CR}}$ are expected to be small with
%respect to $\mu_b b^{\textrm{CR}}$. This $\mu_b$ factor then
which is the product of the MC prediction and a strength parameter:
 $\mu_b b^{\textrm{CR}}$.
% assuming that the contamination from signal
% and other backgrounds in this region is small. 
The $\mu_b$ parameter then
scales the prediction in the SR, $b^{\textrm{SR}}$. Again, the
best-fit values of $\mu$ and $\mu_b$ are obtained by maximizing the
likelihood. 

The predictions from MC simulation which enter the likelihood function
are inherently uncertain, since they rely on assumptions about
theoretical (e.g. renormalization scale) or instrumental (e.g. JES)
parameters. Moreover, the number of events falling into the SR in MC
simulation is also Poisson-distributed
and therefore subject to statistical uncertainties. Because such
parameters are not of fundamental interest in the measurement, they
are called nuisance parameters (NPs), denoted
$\theta_i$. Uncertainties associated with $\theta_i$ are generally computed
independently and this information is incorporated into the likelihood
in two ways. First, the signal and background predictions, denoted
generically as $E$, are written in terms of $\theta_i$:

\begin{equation}
E(\theta_i) = E_0\nu(\theta_i)
%\label{}
\end{equation}

\noindent
where $\nu(\theta_i)$ is chosen {\it a priori} to give $E(\theta_i)$ the
expected $\theta$ dependence for a given source. Additionally, the
uncertainty on $\theta_i$ is incorporated with a response term in the
likelihood function, denoted
$\mathscr{A}(\tilde{\theta_i}\,|\,\theta_i)$, where $\tilde{\theta_i}$ has
been determined to be the best guess for $\theta_i$. In the
frequentist interpretation, $\mathscr{A}(\tilde{\theta_i}\,|\,\theta_i)$
is the sampling distribution for $\theta_i$, and $\tilde{\theta_i}$
is the mean value. 

In the analysis presented in this thesis, three different
$\nu(\theta_i)$ and $\mathscr{A}(\tilde{\theta_i}\,|\,\theta_i)$ are used
in the likelihood function, depending on the
nature of the uncertainty. For both theoretical and instrumental
uncertainties, 

\begin{equation}
\begin{aligned}
\nu(\theta_i) &= (1+\epsilon)^{\theta_i} \\
\mathscr{A}(\tilde{\theta_i}\,|\,\theta_i)
&= \mathcal{G}(\tilde{\theta_i}\,|\,\theta_i,1)
\label{chapter:statistics:equation:nu_theta_A_theta}
\end{aligned}
\end{equation}

\noindent
where $\mathcal{G}(\tilde{\theta_i}\,|\,\theta_i,1)$ is a unit width Gaussian distribution
centered at $\theta_i$. The parameter $\epsilon$ is obtained by
measuring the event yield expectations at their $\pm 1\sigma$
points. Letting $E(1) = (1+\epsilon_{+})^{1}$ and $E(-1) =
(1+\epsilon_{-})^{-1}$, the expectation is written

\begin{equation}
E(\theta) = \left\{
\begin{array}{ll} 
E_0(1+\epsilon_+)^{\theta} & \, \theta\geq 0 \\
E_0(1+\epsilon_-)^{\theta} & \, \theta < 0
\end{array}
\right.
\label{chapter:statistics:equation:E_theta}
\end{equation}

%\begin{equation}
%E(\theta) = \left\{\begin{matrix}
%E_0(1+\epsilon_+)^{\theta} \; \theta\geq0  \\ 
%E_0(1+\epsilon_-)^{\theta} \; \theta<0
%\end{matrix}
%\end{equation}

\noindent
If, for example, the expectation value for an uncertainty source is
3\% greater than the nominal
expectation value $E_0$ at $E(1)$, then $(1+\epsilon_+)^1 = E(1)/E_0 =
1.03$, implying that $\epsilon_+ = 3\%$. Therefore, $\epsilon$
represents the fractional change in the yields due to a given NP. With
the choices in equation~\ref{chapter:statistics:equation:nu_theta_A_theta}, the
sampling distribution for the expectation value $E(\theta)$ is a
log-normal distribution centered at $E_0$. An alternative choice is the same
$\mathscr{A}(\tilde{\theta_i}\,|\,\theta_i)$ with $\nu(\theta) =
(1+\delta\theta)$, which yields an $E(\theta)$ that is normally
distributed about $E_0$ with a width $\sigma = E_0\delta$. If $\theta
< -1/\delta$, the expectation value is negative. To avoid such a
situation, $E(\theta)$ is set to zero in these cases, resulting in a
discontinuity in the expectation value as a function of
$\theta$. For this reason, the log-likelihood sampling distribution is
chosen. Each NP $i$ is then factorized such that $E(\vec{\theta}) =
E_0\prod_i \nu_i(\theta_i)$.

Similar to the statistical uncertainty on the observed number of
data events in the SR, there are statistical uncertainties on the
expected number of events
due to Poisson fluctuations in the MC sample~\cite{bib:Barlow:1993dm}. The expectation value,
$E$, for a given process can be expressed as the product
$\mathscr{L}\cdot \sigma \cdot \epsilon_{\textrm{SR}}$, where
$\mathscr{L}$ is the integrated luminosity, $\sigma$ is the process
cross section, and $\epsilon_{\textrm{SR}}$ is the efficiency for
selecting an event in the SR. Re-expressing the efficiency in terms of
the raw number of MC events selected in the SR and the total number of
events generated, the expectation is

\begin{equation}
E(\vec{\theta}) = \mathscr{L}(\vec{\theta})\cdot \sigma(\vec{\theta}) \cdot \frac{N_{\textrm{SR}}(\vec{\theta})}{N_{\textrm{total}}}
\label{chapter:statistics:figure:expect_factors}
\end{equation}

\noindent
where the dependence on $\vec{\theta}$ has been explicitly
indicated. The random variable $N_{\textrm{SR}}$ is expected to be
Poisson-distributed for the source due to statistical uncertainties,
with a mean of $\hat{N}_{\textrm{SR}}$, and a standard deviation of
$\sqrt{\hat{N}_{\textrm{SR}}}$, where $\hat{N}_{\textrm{SR}}$ is the
measured number of raw events in the SR. Using the fact
that $\mathscr{L}$ and $\sigma$ are constants for this NP,
the ratio of $E(\theta)$ to $E_0$ is equivalent to the ratio of
$N_{\textrm{SR}}$ and $\hat{N}_{\textrm{SR}}$, allowing $E(\theta)$ to
be expressed as a fixed constant multiplying $E_0$:
$E(\theta)= E_0\theta$. With this parameterization, the constraint term
is set to give the {\it a priori}~sampling distribution:

\begin{equation}
\begin{aligned}
\nu(\theta) &= \theta \\
\mathscr{A}(\tilde{\theta}\,|\,\theta) &=
P(\hat{N}_{\textrm{SR}}\,|\,\theta N_{\textrm{SR}})
\end{aligned}
\end{equation}

\noindent
Under these definitions, the most likely value of $E(\theta)$ is $E_0$
and the width $\Delta_E$ is
$E_0/\sqrt{\hat{N}_{\textrm{SR}}}$. To keep the number of NPs to a
minimum, the parameter for the statistical uncertainty is integrated
across all processes in a given bin, yielding a single NP per bin. 

The final type of NP is associated with the statistical uncertainty
due to the number of data events in a control region. Like the
statistical uncertainty from MC predictions, this NP has a constraint
term that is Poisson:

The final type of NP--- the strength parameter associated with a
background that is constrained from a CR--- has been discussed
(equation~\ref{chap:statistics:equation:likelihood_CR}). To be
consistent with the description of other NPs, the constraint term is
written in terms of $\theta$:

\begin{equation}
\begin{aligned}
\nu(\theta) &= \theta \\
\mathscr{A}(\tilde{\theta}\,|\,\theta) &=
P(\tilde{N}_{\textrm{CR}}\,|\, N_{\textrm{CR}}(\theta))
\end{aligned}
\end{equation}

\noindent
Here, $N_{\textrm{CR}}$ is defined as the expected number of
events in the CR: $N_{\textrm{CR}} = \mu s_{\textrm{CR}} + \theta
b_{\textrm{CR}}^{\textrm{target}} + \sum_k
b_{\textrm{CR}}^{\textrm{other}}$. $s_{\textrm{CR}}$ and
$b_{\textrm{CR}}^{\textrm{other}}$ are the expected contaminations
from signal and backgrounds other than the target
background. $\tilde{N}_{\textrm{CR}}$ is the observed number of events
in the CR. In the
likelihood expressions that follow, such NPs are explicitly separated
from the others to illustrate which backgrounds are constrained with
an auxiliary measurement. 

Accounting for the systematic and statistical uncertainties, the
likelihood function can be written with the additional terms outlined
above. Additionally, if the final discriminant in the analysis is
binned in more than one bin, a product over the likelihoods for each
bin is performed. The likelihood in each bin can be written

\begin{equation}
\mathscr{L}_{\textrm{bin }i}(\mu,\vec{\mu}_{b},\vec{\theta}) = P(N_i\,|\,
\mu s(\vec{\theta}) + \sum_j \mu_j b_j(\vec{\theta}) + \sum_k
b_k(\vec{\theta})) \cdot \prod_{j=1}^J P(M_j\,|\, \mu_j
b^{\textrm{CR}}_j(\vec{\theta})) \cdot \prod_{n=1}^K \mathscr{A}(\tilde{\theta_n}\,|\,\theta_n).
\label{chapter:statistics:equation:LH_bin_full}
\end{equation}

\noindent
The first term is the likelihood for bin $i$ of the SR having observed
$N_i$ events. The second term represents the likelihood for the auxiliary
measurement of $M_j$ events in the CR and the final term is the
product of the constraint terms for each of the $K$ NPs considered in
the analysis. All information in the analysis--- the event yields in the SR, NFs from
CR measurements, and systematic uncertainties--- has been reduced to a
single analytic function of the signal strength $\mu$, the NFs
$\vec{\mu^{\prime}}$, and the NPs $\vec{\theta}$. To extract the
best-fit $\mu$, the likelihood is maximized with respect to these
arguments simultaneously using the M{\textsc INUIT}
package~\cite{bib:James:1975dr}. 

