\subsection{Preselection}

With the physics objects defined, a series of pre-selection cuts
designed to isolate the $\ell\nu\ell\nu$ final state is applied. These
cuts are applied in both the VBF analysis and the other jet
bins. First, exactly two leptons with opposite charge are required,
with the leading lepton \pt~$>22$~\gev and the subleading lepton
\pt~$>10$~\gev. These cuts have been optimized to maximize signal
acceptance, while minimizing contamination from background due to jets
faking leptons. A dilepton mass cut of
\mll~$>$~10 (12)~\gev~is applied in the \emme (\eemm) lepton channel,
and to reject background from \ZDY, events with dilepton mass falling
within 15~\gev~of the Z pole mass are cut away. Lepton and jet
kinematic distributions are shown in
figures~\ref{chap:analysis:fig:lead_lep,chap:analysis:fig:sublead_lep,chap:analysis:fig:lead_jet},
and both calo and track \etmiss are shown in figure~\ref{}. The MC models the data
adequately in this phase space region. 

\begin{figure}[h]
\centering
\includegraphics[width=0.45\textwidth]{fig/analysis/preselection_kinematics/CutMll_lepPtLead_mh125_log.eps}
\includegraphics[width=0.45\textwidth]{fig/analysis/preselection_kinematics/CutMll_lepEtaLead_mh125_log.eps}
\caption{The leading lepton \pt~(left) and $\eta$~(right) at
  pre-selection for all lepton flavors combined. The error band
  includes statistical uncertainties only.}
\label{chap:analysis:fig:lead_lep}
\end{figure}

\begin{figure}[h]
\centering
\includegraphics[width=0.45\textwidth]{fig/analysis/preselection_kinematics/CutMll_lepPtSubLead_mh125_log.eps}
\includegraphics[width=0.45\textwidth]{fig/analysis/preselection_kinematics/CutMll_lepEtaSubLead_mh125_log.eps}
\caption{The sub-leading lepton \pt~(left) and $\eta$~(right) at
  pre-selection for all lepton flavors combined. The error band
  includes statistical uncertainties only.}
\label{chap:analysis:fig:sublead_lep}
\end{figure}

\begin{figure}[h]
\centering
\includegraphics[width=0.45\textwidth]{fig/analysis/preselection_kinematics/CutMll_jetPtLead_mh125_log.eps}
\includegraphics[width=0.45\textwidth]{fig/analysis/preselection_kinematics/CutMll_jetEtaLead_mh125_lin.eps}
\caption{The leading jet \pt~(left) and $\eta$ (right) at pre-selection (\mll) for all lepton flavours combined. The
error band includes statistical uncertainties only.}
\label{chap:analysis:fig:lead_jet}
\end{figure}

%\begin{figure}[h]
%\centering
%\includegraphics[width=0.45\textwidth]{fig/analysis/preselection_kinematics/CutMll_MET_TrackHWW_Clj_mh125_log.eps}
%\includegraphics[width=0.45\textwidth]{fig/analysis/preselection_kinematics/CutMll_METRel_TrackHWW_Cl_mh125_log.eps}
%\caption{The \jcorrptmiss (left) and \jcorrptmissrel (right)
%  distributions at the pre-selection cut stage (\mll) for all lepton
%  flavours combined. No systematic uncertainty is \
%displayed.}
%\label{chap:analysis:fig:met}
%\end{figure}

\subsection{VBF-specific pre-selection}
\label{subsec:vbf_select}

The VBF analysis selection departs from the 0,1j selection starting
with the \etmiss cuts. Looking at the jet multiplicity distributions
after pre-selection (figure XX), the dominant backgrounds are \ttbar~in
both lepton channels and also \ZDY in the \eemm channel. In the \emme
channel, although signal has real \etmiss from the neutrinos in the
$W$ decays, there is not an \etmiss cut applied, due to the nature of
the backgrounds which also contain real \etmiss. Without a cut on
\etmiss, there is a gain in signal acceptance corresponding to an expected
significance gain of 6\% compared to the cut value of 20~\gev~in the previous
analysis~\cite{bib:hww_moriond_2013}. In the \eemm channel, on the other hand, two cuts are
applied: track \etmiss~$>40$~\gev~and calo \etmiss~$>45$~\gev. These
cuts suppress \ZDY, a process with ``fake'' \etmiss that arises from
the mismeasurement of jets and soft tracks. 

Additionally, the number of jets is required to be greater than or
equal to two to pick out the tag jets in the VBF process. To suppress
top background, events with one or more $b$-tagged jets are vetoed
(BJV), a cut that removes 94\% of \ttbar~and 87\% of ST, while
retaining \textapprox{70\%} of signal. The remaining top background
consists of heavy flavor jets that fall outside of the tracking volume for
$b$-tagging, jets within the tracking volume that are not tagged due
to inefficiency in the algorithm, or ISR jets from gluons. Jet pairs
that arise from QCD vertices and are kinematically similar to the tag jets in VBF, i.e. a large
rapidity gap, have more soft QCD activity at central rapidity due to
color exchange. The central jet veto (CJV) exploits this by requiring
that the highest \pt~jet that falls between the two leading jets has
\pt~$<20$~\gev. This cut retains events without any central jets
because the lead \pt~of the central jets is set to a value of -1 in
this case. In addition to the CJV, an outside lepton veto (OLV) is
applied, exploiting the tendency for leptons resulting from the top decay,
$t\rightarrow{Wb}\rightarrow{\ell\nu b}$, to fall outside of the
pseudorapidity of the jets. The OLV is defined such that if either of
the leptons falls outside of the pseudorapidity gap, the event is
vetoed. 

