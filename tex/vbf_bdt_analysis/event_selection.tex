
With the physics objects defined, event-level selection
cuts are applied sequentially in order to enhance signal. These cuts
can be categorized into those which are shared with the ggF analyses,
and those which are specific to the VBF analysis.

\subsection{Common Preselection}

For all \hwwlnln analyses, exactly two leptons with opposite charge
are required, with the leading lepton $\pt>22 \gev$ and the subleading lepton
$\pt>10 \gev$. These cuts have been optimized to maximize signal
acceptance, while minimizing contamination from background due to jets
faking leptons. A dilepton mass cut of
$\mll> 10$ $(12) \gev$ is applied in the \emme (\eemm) lepton channel,
and to reject background from \ZDY in the \eemm channel, events with dilepton mass falling
within $15 \gev$ of the $Z$ pole mass are cut away. The \pt~and $eta$
distributions of the lepton (jet) with the highest \pt~in the event--
called the leading lepton (jet)-- are shown in
figure~\ref{chap:analysis:fig:lep_lead}
(\ref{chap:analysis:fig:jet_lead}) after the common pre-selection cuts
have been applied. MC simulation models the data
adequately in this phase space region. Discrepancies in the jet
distributions at high \pt~ and $|\eta|$ which lie outside of the statistical uncertainty band
are covered by the systematic uncertainties associated with the JES
and JER calibrations (section~\ref{chap:reco:sec:jet}). 

\begin{figure}[h]
\centering
\subfigure[Transverse momentum]{
\includegraphics[width=0.45\textwidth]{fig/analysis/preselection_kinematics/CutMll_lepPtLead_mh125_log.eps}
\label{chap:analysis:fig:lep_lead_pt}
}
\subfigure[Pseudo-rapidity]{
\includegraphics[width=0.45\textwidth]{fig/analysis/preselection_kinematics/CutMll_lepEtaLead_mh125_log.eps}
\label{chap:analysis:fig:lep_lead_eta}
}
\caption{The leading lepton~\subref{chap:analysis:fig:lep_lead_pt}
  \pt~and~\subref{chap:analysis:fig:lep_lead_eta}~$\eta$ at
  pre-selection for all lepton flavors combined. Data-driven
  corrections to backgrounds are not applied at this stage. The error band
  includes statistical uncertainties only.}
\label{chap:analysis:fig:lep_lead}
\end{figure}

%\begin{figure}[h]
%\centering
%\includegraphics[width=0.45\textwidth]{fig/analysis/preselection_kinematics/CutMll_lepPtSubLead_mh125_log.eps}
%\includegraphics[width=0.45\textwidth]{fig/analysis/preselection_kinematics/CutMll_lepEtaSubLead_mh125_log.eps}
%\caption{The sub-leading lepton \pt~(left) and $\eta$~(right) at
%  pre-selection for all lepton flavors combined. The error band
%  includes statistical uncertainties only.}
%\label{chap:analysis:fig:sublead_lep}
%\end{figure}

\begin{figure}[h]
\centering
\subfigure[Transverse momentum]{
\includegraphics[width=0.45\textwidth]{fig/analysis/preselection_kinematics/CutMll_jetPtLead_mh125_log.eps}
\label{chap:analysis:fig:jet_lead_pt}
}
\subfigure[Pseudo-rapidity]{
\includegraphics[width=0.45\textwidth]{fig/analysis/preselection_kinematics/CutMll_jetEtaLead_mh125_lin.eps}
\label{chap:analysis:fig:jet_lead_eta}
}
\caption{The leading jet~\subref{chap:analysis:fig:jet_lead_pt}
  \pt~and~\subref{chap:analysis:fig:jet_lead_eta} $\eta$ at
  pre-selection for all lepton flavors combined. Data-driven
  corrections to backgrounds are not applied at this stage. The
error band includes statistical uncertainties only.}
\label{chap:analysis:fig:jet_lead}
\end{figure}

%\begin{figure}[h]
%\centering
%\includegraphics[width=0.45\textwidth]{fig/analysis/preselection_kinematics/CutMll_MET_TrackHWW_Clj_mh125_log.eps}
%\includegraphics[width=0.45\textwidth]{fig/analysis/preselection_kinematics/CutMll_METRel_TrackHWW_Cl_mh125_log.eps}
%\caption{The \jcorrptmiss (left) and \jcorrptmissrel (right)
%  distributions at the pre-selection cut stage (\mll) for all lepton
%  flavours combined. No systematic uncertainty is \
%displayed.}
%\label{chap:analysis:fig:met}
%\end{figure}

\subsection{VBF-specific pre-selection}
\label{subsec:vbf_select}

The VBF analysis selection departs from the common selection starting
with a cut on the number of jets in the event, which is required to be
greater than or equal to two, with the two leading jets considered the
tag jets associated with VBF. Looking at the jet multiplicity
distributions after pre-selection (figure~\ref{chap:analysis:fig:jet_mult}), the
dominant backgrounds with $\Njet \geq 2$ are \ttbar~in
both lepton channels and also \ZDY in the \eemm channel. Additional
pre-selection cuts are designed to suppress these large backgrounds. 

\begin{figure}[h]
    \centering
    \subfigure[\emme channel]{
    \includegraphics[width=0.45\textwidth]{analysis/emme_CutMET_VBF_forPlots_m_jet_n_mh125_lin.eps}
    \label{chap:analysis:fig:jet_mult_df}
    }
    \subfigure[\eemm channel]{
    \includegraphics[width=0.45\textwidth]{analysis/eemm_CutMET_VBF_forPlots_m_jet_n_mh125_lin.eps}
    \label{chap:analysis:fig:jet_mult_sf}
    }
    \caption{Jet multiplicity distributions
      for~\subref{chap:analysis:fig:jet_mult_df} \emme channel
      and~\subref{chap:analysis:fig:jet_mult_sf} \eemm channel after
      common pre-selection. In~\subref{chap:analysis:fig:jet_mult_sf}
    \etmiss cuts and $Z$ veto are also applied. Error band represents
    statistical uncertainties.}
\label{chap:analysis:fig:jet_mult}
\end{figure}

In the \emme channel, although signal has \etmiss from the neutrinos in the
$W$ decays, there is not an \etmiss cut applied, due to the nature of
the backgrounds which also contain \etmiss. Without a cut on
\etmiss, there is a gain in signal acceptance corresponding to an expected
significance gain of 6\% compared to the cut value of $20 \gev$ in the previous
analysis~\cite{bib:hww_moriond_2013}. 

Calorimeter and track \etmiss distributions in the \eemm channel after the $\Njet \geq 2$
cut are shown in figure~\ref{met_preselect}. \ZDYll lies at low \etmiss,
as this process does have neutrinos in the final state. Therefore, to reject this
background, \etmiss cuts are applied, with the optimal cuts being
$\trkmet>40 \gev$ and $\calomet >45 \gev$.

\begin{figure}[h]
    \centering
    \subfigure[Calorimeter \etmiss]{
    \includegraphics[width=0.45\textwidth]{analysis/preselection_kinematics/met_plots_old_style/eemm_Cut_TwoJet_noMET_caloMET_mh125_lin.eps}
    \label{chap:analysis:fig:met_calo}
    }
    \subfigure[Track \etmiss]{
    \includegraphics[width=0.45\textwidth]{analysis/preselection_kinematics/met_plots_old_style/eemm_Cut_TwoJet_noMET_trackMET_mh125_lin.eps}
    \label{chap:analysis:fig:met_track}
    }
    \caption{Distributions of~\subref{chap:analysis:fig:met_calo}
      \calomet and~\subref{chap:analysis:fig:met_track} \trkmet in
      the \eemm channel with the common pre-selection cuts and $\Njet
      \geq 2$. Data-driven background corrections are not applied at
      this stage, and the error band represents statistical
      uncertainties only.}
\label{chap:analysis:fig:met_preselect}
\end{figure}

To suppress top background, events with one or more $b$-tagged jets
are vetoed (BJV), a cut that removes 94\% of \ttbar~and 87\% of ST, while
retaining \textapprox{70\%} of signal. The remaining top background
consists of heavy flavor jets that fall outside of the tracking volume for
$b$-tagging, jets within the tracking volume that are not tagged due
to inefficiency in the algorithm, or ISR jets from gluons. 

Jet pairs that arise from QCD vertices and are kinematically similar
to the tag jets in VBF, i.e. a large
rapidity gap, have more soft QCD activity at central rapidity due to
color exchange. The central jet veto (CJV) exploits this by requiring
that the highest \pt~jet that falls between the two leading jets has
$\pt<20 \gev$. This cut retains events without any central jets
because the lead \pt~of the central jets is set to a value of -1 in
this case. In addition to the CJV, an outside lepton veto (OLV) is
applied, exploiting the tendency for leptons resulting from the top decay,
$t\rightarrow{Wb}\rightarrow{\ell\nu b}$, to fall outside of the
pseudorapidity of the jets. The OLV is defined such that if either of
the leptons falls outside of the pseudorapidity gap, the event is
vetoed. 

The final preselection cut rejects \Ztautaunody processes,
which contribute in the signal region if both $\tau$s decay
leptonically. If the tauon decay products have momenta which is
far less than their mass, then to a good approximation, the direction
of the tauons is given by the direction of the reconstructed
leptons. The invisible products of the tauon decays are assumed to
have \pt~given by the \etmiss, allowing the tauon momentum to be
reconstructed. From this, the invariant mass of the $\tau\tau$ system
is computed, and \Ztautaunody background is rejected by
requiring that $\mtt < m_Z - 25 \gev$. 

