
Although \wjets and QCD multijet processes (figure~\ref{}) do not contain two
leptons in the final state, they are expected to contribute in the
signal region in cases where a jet is mis-reconstructed as a prompt
lepton.  Such leptons are considered ``fake'', whether they are
mis-reconstructed from charged hadron tracks or from heavy hadron
decays to true non-prompt leptons. Given that this background is due
to instrumental effects, a reliable estimate can only be obtained with
data-driven methods. In contrast to the data-driven estimates
described above, the procedure described in the following section
is purely data-driven in the sense that both the normalization and the
kinematic shapes are derived from a control region. In other words,
\wjets MC templates are not used at all.

The \wjets CR is the same as the analysis signal region except that
the object selection has been adjusted to enrich the region with fake
leptons. This is accomplished by loosening the quality criteria for
one of the leptons and requiring that it does not pass the analysis-level
lepton selection, while the other lepton is required to pass. The
former lepton type is referred to as ``anti-identified''. For
anti-identified electrons, the calorimeter isolation requirement is
loosened to $\et^{R=0.3}/\et<0.30$, the track isolation requirement
is $p_T^{R=0.3}/\et<0.16$, the conversion flag and $b$-layer
requirements are removed and the electron is required to fail the {\it
  medium} identification requirement. For muons, the calorimeter
isolation requirements are also loosened, the track isolation
requirements are completely removed, and the transverse impact parameter
significance cut is removed.

To extrapolate from the \wjets CR to the SR, first the non-\wjets
background is subtracted, and then an extrapolation factor, called the
fake factor, is applied. The fake factor is defined as

\begin{equation}
\label{chap:analysis:equation:fake_factor}
f_{\textrm{fake}} =
\frac{N_{\textrm{id}}}{N_{\textrm{anti-id}}}
\end{equation}

\noindent
where $N_{\textrm{id}}$ is the number of jets that pass the
full lepton selection and $N_{\textrm{anti-id}}$ is the number
of jets which pass the anti-identified selection. \fakefact is
evaluated with jets in a jet-rich $Z$ CR in bins of lepton \pt~and
$\eta$. From the fake factor, the \wjets estimate in the SR is given
by

\begin{equation}
\label{chap:analysis:equation:wjets_est}
N_{id+id}^{\wjets} = \fakefact\cdot (N_{\textrm{id+anti-id}} -
N_{\textrm{id+anti-id}}^{\textrm{QCD}} -
N_{\textrm{id+anti-id}}^{\textrm{non-fake,MC}})
\end{equation}

\noindent
where $N_{\textrm{id+anti-id}}$ is the total number of events in the
\wjets CR, $N_{\textrm{id+anti-id}}^{\textrm{QCD}}$ is data-driven QCD
multijet estimate in the \wjets CR, which is described below, and
$N_{\textrm{id+anti-id}}^{\textrm{non-fake,MC}}$ is the MC prediction
for the non-fake contribution in the \wjets CR. 

The dijet control region is similar to the \wjets control region,
except that instead of the requirement of one anti-identified lepton,
both of the leptons satisfy this requirement. The fake factors for
this background are obtained using the same procedure as \wjets,
except that the control region is rich in QCD dijets. Because there
are two fake leptons, two fake factors are derived, where the second
fake factor accounts for bias introduced with the requirement of an
additional anti-identified lepton. The QCD estimate in the signal
region is then expressed as

\begin{equation}
N_{id+id}^{\textrm{QCD}} = \fakefact^{\prime\prime}\cdot\fakefact^{\prime}\cdot (N_{\textrm{anti-id+anti-id}} -
N_{\textrm{anti-id+anti-id}}^{\wjets\textrm{,MC}} -
N_{\textrm{anti-id+anti-id}}^{\textrm{non-fake,MC}})
\end{equation}

\noindent
where $\fakefact^{\prime\prime}$ and $\fakefact^{\prime}$ are the two
dijet fake factors, $N_{\textrm{anti-id+anti-id}}$ is the total number
of data events in the dijet CR,
$N_{\textrm{anti-id+anti-id}}^{\wjets\textrm{,MC}}$ is the \wjets
contamination predicted by simulation in the CR, and
$N_{\textrm{anti-id+anti-id}}^{\textrm{non-fake,MC}}$ is the non-fake
contamination. 

To estimate the QCD contamination in the \wjets CR
($N_{\textrm{id+anti-id}}^{\textrm{QCD}}$ in
equation~\ref{chap:analysis:equation:wjets_est}), there is an
extrapolation from the dijet CR to the \wjets CR, given by

\begin{equation}
N_{id+anti-id}^{\textrm{QCD}} = 2\cdot\fakefact^{\prime\prime}\cdot (N_{\textrm{anti-id+anti-id}} -
N_{\textrm{anti-id+anti-id}}^{\wjets\textrm{,MC}} -
N_{\textrm{anti-id+anti-id}}^{\textrm{non-fake,MC}})
\end{equation}

\noindent
where the factor of two arises due to the fact that events with two
anti-identified leptons can enter the \wjets CR if either lepton is
identified. 

Backgrounds due to fakes are relatively small in the BDT signal
region. The primary reason for this is that the jets in \wjets and QCD
processes tend to be more central in rapidity, but the BDT selects a
region of phase space with forward jets. The \wjets estimate in the
\emme channel relative to the total background prediction is 2\%, 3\%,
and 0\%, in the respective BDT fit bins and the QCD estimate is even
smaller-- 0.4\%, 0\%, and 0\%. 

\noindent Systematics go here or later?

