
The ATLAS calorimeter system, described in section~\ref{chap:lhc_atlas:sec:calorimeter}, is crucial
for reconstructing electrons, photons, jets, and missing transverse
energy. Just as tracks are reconstructed from hits in the ID,
calorimeter cells are grouped into clusters that represent the
transverse energy of incident particles. With the energy and direction
information from the clusters, it is possible to reconstruct the
4-momentum of a particle. Moreover, cluster shapes are used to identify
the particles in the collision. Two types of clustering algorithms are
used in ATLAS: sliding window and topological clustering~\cite{bib:Lampl:2008zz}. The former
is effective in the reconstruction of electrons, photons, and hadronic
$\tau$ decays, while the latter is most performant for jets and
missing transverse energy.

\subsection{Sliding window clustering}
\label{chap:reconstruction:sec:cluster:subsec:sliding_window}

In sliding window clustering, calorimeter cells in a fixed rectangular
$\eta-\phi$ window are summed, and clusters are built around windows
for which \et~is maximized. The first step is to partition the
calorimeter into an $\eta-\phi$ grid with each grid element having dimension
$\Delta{\eta}\times \Delta{\phi} = 0.025 \times 0.025$. For each grid
element, the enclosed cell energies are summed across the calorimeter layers,
yielding a ``tower'', the object from which clusters are
constructed. A window of area $\eta_{\textrm{window}} \times
\phi_{\textrm{window}}$ is scanned across the grid of towers, and at
each point, the transverse energy sum of all towers within the window
is computed. If a local maximum above an energy threshold is found, the location associated with
this point is computed as the weighted vector sum of the tower
energies, and the resulting $(\eta,\phi)$ coordinate is used to seed the final step in which the
clusters are built. If two of these seeds fall within two $\eta$ or
two $\phi$ grid units of eachother, the seed with the lower \et~is
discarded. 

Calorimeter cells are assigned to a cluster if they are enclosed by the
rectangle of size $\eta_{\textrm{cluster}} \times
\phi_{\textrm{cluster}}$ centered at the seed defined in the previous
step. The dimensions of the cluster rectangle as well as the seed
location depend on the calorimeter layer. Furthermore,
$\eta_{\textrm{cluster}}$ and $\phi_{\textrm{cluster}}$ vary depending on
the predicted particle type. In the barrel region, the window size for electrons and
converted photons is $3 \times 7$ in units of $\Delta{\eta}\times
\Delta{\phi}$. Because electrons bend in the $\phi$ direction in the ID
$B$-field and emit {\it Bremsstrahlung} photons, electron calorimeter
showers tend to be elongated in the $\phi$ direction. Similarly,
photons that convert to electron-positron pairs also have showers that
are elongated in $\phi$, and therefore the window is set to $3 \times
7$. Unconverted photons, on the other hand, tend to shower in a
smaller $\phi$ region, and consequently the window dimension is $3
\times 5$. 

\subsection{Topological clustering}
\label{chap:reconstruction:sec:cluster:subsec:topo}

In contrast to sliding window clustering, the topological clustering
algorithm iteratively adds neighboring cells to seed cells, resulting in variable
size clusters. Seed cells are calorimeter cells with a signal-to-noise
ratio that falls above a large threshold $t_{\textrm{seed}}$. Signal
is defined as the energy deposited in the cell for the EM calorimeter,
and as the absolute value of the energy in the hadronic
calorimeter. There are two sources of noise: readout electronics and
pileup. Noise from the electronics is computed based on the cell
gain, while the expected noise to due pileup is estimated from the
beam conditions, and the two components are added in quadrature. The
noise in hadronic calorimeter cells in the outermost barrel layer as a
function of the mean number of interactions per bunch crossing is
shown in figure~\ref{chap:reconstruction:fig:tile_noise}. In these
cells, the pileup noise increases fairly linearly with $\left \langle \mu \right \rangle$.

\begin{figure}[h]
    \centering
    \includegraphics[width=0.6\textwidth]{fig/reconstruction/noise_D5D6_8TeV_25ns_newLVPS.eps}
    \caption[]{Noise in hadronic calorimeter cells located in the
    outermost layer in $0.9 < |\eta| < 1.3$ as a function of
    $\left \langle \mu \right \rangle$. Collision data at $\rts = 7 \tev$ with 25 ns bunch spacing
    is shown in blue and red. Data points at $\left \langle \mu \right \rangle > 20$ are from
    simulation. The dashed line is the expected noise from
    electronics.\cite{bib:Araque:1561207}}
\label{chap:reconstruction:fig:tile_noise}
\end{figure}

Seed cells, called proto-clusters, are ranked according to $S/N$ and for
each seed, if a neighboring cell passes another $S/N$ threshold,
$t_{\textrm{neighbor}}$, the neighbor is added to the
proto-cluster. In cases where the neighboring cell is adjacent to two
nearby seeds, the two proto-clusters are merged. In the next
iteration, the neighbors with $S/N > t_{\textrm{neighbor}}$ become the
new seeds, and the procedure is repeated. Neighboring cells
which pass a threshold $t_{\textrm{cell}}$, but fall below
$t_{\textrm{neighbor}}$, are added to the proto-cluster, but do not
become seeds in the subsequent iteration. These low $S/N$ cells are
retained to better model the tails of the calorimeter shower. Cluster
building continues until the list of seeds is exhausted. Because
neighboring cells include those in adjacent calorimeter layers in
addition to those in the $\eta-\phi$ plane, the
resulting topological clusters are three-dimensional objects which
can span across the EM and hadronic calorimeters. 

For events with low particle multiplicity, the clustering algorithm
described above is sufficient. However, due to the high energy density
of actual ATLAS events, the above algorithm creates clusters
that span large regions of the calorimeter. To resolve individual
particles, the clusters are split into sub-clusters, using local energy
maxima within the clusters. If a cell has $E > 400$~\mev, this energy
is greater than that of the neighboring cells, and four or more
neighbors have $S/N$ above a threshold, the cell is added to a list of
seeds. These seeds are then used to grow clusters with the same
algorithm in the previous step, except that only the cells that fall
within the parent cluster are added, no threshold is applied, and
overlapping clusters are not merged, i.e. cells can be associated with
two clusters, and the energy of the cell is shared between the two
clusters. The resulting collection of clusters is combined with
the parent clusters without local maxima to form the final list of
topological clusters in the event. 

There are two topological cluster types in ATLAS, defined by
the $S/N$ thresholds for cluster building. The ``electromagnetic 633''
(``combined 420'') cluster type has $t_{\textrm{seed}}=6~(4)$,
$t_{\textrm{neighbor}}=3~(2)$, and $t_{\textrm{seed}}=3~(0)$. Electromagnetic
633 clusters only use cells from the EM calorimeter, while combined
420 clusters can include cells from either calorimeter. With higher
$S/N$ thresholds, the rate at which noise results in a cluster is
smaller for EM 633 clusters. Combined 420 clusters, on the other hand,
have been optimized to find low energy deposits, with sufficient noise
rejection power. 
