
Prior to the summer of 2012, the gauge structure of the SM had held up
to repeated experimental tests, with the exception of one crucial
part--- evidence of the Higgs boson associated with $SU(2) \times
U(1)$ symmetry breaking. This long sought after particle had been one
of the primary motivations for the construction of the LHC, a $pp$
collider with a center-of-mass energy (\sqrts) expected to be large
enough to observe the particle. Following the discovery of the Higgs
boson in July 2012, LHC experiments entered a new phase of precision
measurements of the Higgs couplings. In the following section, Higgs
physics at the LHC is discussed.

The introduction of the Higgs field into the SM Lagrangian gives rise
to a consistent set of testable predictions. The Higgs boson should
behave like a chargeless spin-0 particle. It should couple to weak
gauge bosons through terms of the form $HVV$ ($V=W/Z$), with a strength
proportional to the square of the mass of the gauge boson. Moreover, as a
consequence of fermion mass generation, the Higgs boson should couple
to fermions with a strength that grows linearly their masses. Because the
masses of the gauge bosons and fermions are well-measured, the Higgs
couplings are determined. In fact, the only remaining parameter to which
Higgs predictions are sensitive is the mass of the Higgs boson $m_H$,
which is not constrained by the theory itself. It is, however,
argued that, if the theory is to remain unitary and have a stable vacuum,
the Higgs mass should lie in the range $50 \gev \lesssim m_H \lesssim
800 \gev$~\cite{bib:Djouadi:2005gi}. Since the Higgs boson is a
massive particle that couples to both fermions and gauge bosons, it is
found in loop diagrams at higher orders in perturbation theory. These
diagrams contribute non-negligible corrections to SM input parameters
that are observable in precision electroweak measurements. Using such measurements
from the LEP, SLC, and Tevatron experiments, $m_H$ had been
constrained to be $91^{+30}_{-23} \gev$ at the one standard deviation
level prior to the discovery~\cite{bib:Baak:2011ze}. \note{Show $\chi^2$
figure?}. A direct search done at LEP2 placed a lower bound of $m_H >
114.4 \gev$ at the 95\% confidence level. These experimental
constraints helped to guide the LHC experiments prior to the Higgs
discovery. Following the discovery, the Higgs mass has been measured
to be $125.7 \pm 0.4$ by the CMS collaboration~\cite{bib:CMS-PAS-HIG-13-005} and $125.36 \pm
0.41$ by the ATLAS collaboration~\cite{bib:Aad:2014aba}. 

\begin{figure}[h]
    \centering
    \subfigure[ggF]{
    \includegraphics[width=0.45\textwidth]{fig/theory/ggf_diag.pdf}
    \label{chap:theory:fig:higgs_diag_ggf}
    }
    \subfigure[VBF]{
    \includegraphics[width=0.40\textwidth]{fig/theory/vbf_diag.pdf}
    \label{chap:theory:fig:higgs_diag_vbf}
    }
    \subfigure[$VH$]{
    \includegraphics[width=0.45\textwidth]{fig/theory/vh_diag.pdf}
    \label{chap:theory:fig:higgs_diag_vh}
    }
    \subfigure[$t\bar{t}H$]{
    \includegraphics[width=0.40\textwidth]{fig/theory/tth_diag.pdf}
    \label{chap:theory:fig:higgs_diag_tth}
    }
    \caption[]{}
\label{chap:theory:fig:higgs_diagrams}
\end{figure}

The dominant mechanisms for the production of a Higgs boson at
a $pp$ collider, shown in terms of Feynman diagrams in
figure~\ref{chap:theory:fig:higgs_diagrams}, are dictated by the fact
that the Higgs boson preferentially couples to more massive particles. The
process with the
largest cross section is gluon-gluon fusion (ggF), $gg\rightarrow{H}$, characterized by
two incoming gluons that effectively couple to the Higgs through a
quark loop. Only the heavy top and bottom quarks contribute in this loop. If the
center-of-mass energy is large with respect to $m_H$, then the Higgs
can be produced with a small fraction of the incoming proton
momentum, or in the $x$ region where the gluon pdf is
large (figure~\ref{chap:theory:fig:pdf_set}). Higgs production by direct coupling between the Higgs
and incoming quarks is suppressed by the pdfs for the heavy quarks for
which there is a non-negligible coupling to the Higgs.

The second largest Higgs production mechanism at $pp$ colliders is the
vector boson fusion (VBF) process, whereby two incoming quarks radiate
virtual weak gauge bosons that then fuse to form the Higgs,
$qq\rightarrow{q^{\prime}q^{\prime}V^{\ast}V^{\ast}}\rightarrow{q^{\prime}q^{\prime}H}$. Though
this process can proceed via either $W$ or $Z$ fusion, the
contribution of the $W$ diagram is around three times that of the $Z$,
due to the fact that $W$ bosons couple more strongly to
fermions~\cite{bib:Djouadi:2005gi}. In spite of the smaller cross
section, this process is a powerful probe of the Higgs sector due to its
characteristic final state. Since the energy of the radiated weak
bosons is significantly less than that of the incoming quarks, the
deflection of these quarks is small, and therefore the outgoing quarks
manifest as high energy forward jets. Another feature of these events is that
there is little QCD radiation between the outgoing quarks, due to the
absence of the flow of color between them. These two characteristics
allow such Higgs events to be efficiently isolated from the background
events, which, in hadron colliders, are likely to include central
jets. 

Associated production of the Higgs with either weak gauge bosons or
top quarks is also visible at the LHC. The former occurs when the initial state quarks form an
off-shell gauge boson that then splits into a Higgs and a gauge boson:
$qq\rightarrow{V^{\ast}\rightarrow{VH}}$. The latter process is
similar to ggF in that an effective coupling between gluons and the
Higgs is mediated through top quarks, but in this case the quarks
appear as outgoing particles: $gg\rightarrow{t\bar{t}H}$. 

\begin{table}
\begin{center}
\renewcommand{\arraystretch}{1.2}
%\resizebox{0.8\textwidth}{!}{
    \begin{tabular}{ l | c }
    \hline
    Higgs Decay & Branching Fraction \\
    \hline \hline
    $b\bar{b}$ & 0.571 \\
    $WW$ &  0.221 \\
    $gg$ &  $8.53 \times 10^{-2}$ \\
    $\tau\tau$ & $6.25 \times 10^{-2}$ \\
    $c\bar{c}$ &  $2.88 \times 10^{-2}$ \\
    $ZZ$ &  $2.74 \times 10^{-2}$ \\
    $\gamma\gamma$ &  $2.28 \times 10^{-3}$ \\
    $Z\gamma$ &  $1.57 \times 10^{-3}$ \\
    $\mu\mu$ &  $2.17 \times 10^{-4}$ \\
    \hline
    \end{tabular}
%}
\caption[]{\cite{bib:CERNYellowReportPageBR3}}
\label{chap:theory:tab:higgs_br}
\end{center}
\end{table}

The cross sections of these four production processes at $\sqrts =
8 \tev$ are shown in
figure~\ref{chap:theory:fig:higgs_sigma}\cite{bib:Dittmaier:2011ti}. Across
the $m_H$ range shown, the ggF process cross section is approximately
an order of magnitude larger than that of VBF, and at the measured
$m_H$ of $125.4 \gev$, $\sigma_{\textrm{ggF}} = 19.15$~pb and
$\sigma_{\textrm{VBF}} =
1.573$~pb~\cite{bib:CERNYellowReportPageAt8TeV}. The $WH$ ($ZH$) cross
section is 0.6970 pb (0.4112 pb), and $\sigma_{t\bar{t}H} = 0.1280$,
about two orders of magnitude smaller than $\sigma_{\textrm{ggF}}$. 

\begin{figure}[h]
    \centering
    \subfigure[Higgs production cross section.]{
    \includegraphics[width=0.45\textwidth]{fig/theory/Higgs_XS_8TeV_LM200.eps}
    \label{chap:theory:fig:higgs_sigma}
    }
    \subfigure[Higgs decay branching fractions.]{
    \includegraphics[width=0.45\textwidth]{fig/theory/Higgs_BR_LM.eps}
    \label{chap:theory:fig:higgs_bf}
    }
    \caption[Higgs boson production cross section and branching
      fraction as a function of the Higgs mass parameter.]{Higgs boson
    production cross section and branching
      fraction as a function of the Higgs mass parameter~\cite{bib:Dittmaier:2011ti}.}
\label{chap:theory:fig:higgs_sigma_bf}
\end{figure}

Once produced, the unstable Higgs boson decays instantaneously into the
particles to which it couples. The probability for decaying into a
given set of particles is quantified by the branching fraction ($\mathscr{B}$), shown
for the allowed Higgs decays in
figure~\ref{chap:theory:fig:higgs_sigma_bf}. In the $m_H < 130 \gev$
region, the dominant decays are to two bottom quarks, gluons, and tau
leptons. Above this mass, the branching ratios for decays into $WW$ and $ZZ$
become dominant as $m_H$ nears the threshold for the decay into
on-shell $W$s and $Z$s. At the measured Higgs mass, the dominant
decays, ordered by $\mathscr{B}$, are $b\bar{b}$, $WW$, $gg$,
$\tau\tau$, $c\bar{c}$, and $ZZ$. Though there is no direct coupling to
$\gamma\gamma$ and $Z\gamma$, the Higgs can decay to these particles
through a heavy quark or a $W$ boson loop. These two decays, along
with $H\rightarrow{\mu\mu}$, are the remaining three decays at $m_H =
125.4 \gev$. The exact values are shown in
table~\ref{chap:theory:tab:higgs_br}. 

Scattering experiments that aim to measure the Higgs boson are guided
by the above SM predictions for Higgs cross sections and branching
fractions. In principle, the best experimental sensitivity is achieved
by isolating the dominant production mechanism and decay; however, in
practice, this is not always feasible due to experimental
considerations. Hadron colliders produce enormous backgrounds from QCD
processes with final states consisting of quarks and gluons. The Higgs
process that is produced at the highest rate at $m_H =
125.4 \gev$, $gg\rightarrow{H}\rightarrow{b\bar{b}}$, lies in a region
of phase space that is saturated by irreducible QCD
background, making it impossible to observe this Higgs decay. To
suppress QCD backgrounds, final states are required to have photons or
at least one charged lepton. The most sensitive decay channels at the
LHC are $\gamma\gamma$, $ZZ^{(\ast)}\rightarrow{\ell\ell\ell\ell}$,
and $\wwlnln$. The former two channels compensate for a smaller
$\mathscr{B}$ because the final state particles allow the resonant
peak of the Higgs to be resolved against the background. The $\wwlnln$
channel, on the other hand, has final state neutrinos that are not
detectable, making the Higgs mass peak harder to detect against
background. 
