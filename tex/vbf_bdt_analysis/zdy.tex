
In the \eemm channel, the dominant background, \ZDY, is significantly
suppressed by requiring the events to fall at high values of
\etmiss. Processes which produce a $Z/\gamma^{*}$ in association with
jets do not have ``real'' \etmiss from invisible final state
particles. Instead, \etmiss arises due to the mis-measurement of
leptons or jets in the final state, as well as soft QCD activity from the underlying
event or pile-up. Such instrumental effects are difficult to model in
MC simulation, motivating the use of data-driven approaches. 

To correct for mis-modeling of calorimeter-based \etmiss in \ZDY MC, the
efficiency of the \calomet$>45$~\gev~cut is taken from data. After
applying the common preselection cuts and the track \etmiss cut, the
\mll-\etmiss plane is partitioned into four orthogonal regions, labeled A, B, C,
and D (table~\ref{chap:analysis:tab:bdt_abcd_cartoon}). Regions C and D fall in the $Z$ peak ($|\mll -
m_Z| < 15$~\gev), with region C at high \etmiss ($>45$~\gev) and
region D at low \etmiss ($25 < $~\gev~$\etmiss$~$ <
45$~\gev). Similarly, regions A and B fall at high and low \etmiss,
respectively, but at low \mll (\mll$< 75$~\gev). In this way, region A
corresponds to the \eemm SR. Region B is rich in \ZDY, with
\textapprox{5$\%$} contamination from other backgrounds. Consequently,
data from this control region is used to obtain the \ZDY template in
region A. To extrapolate from region B to region A, the number of
events is scaled by the \etmiss efficiency from regions C and D. The
estimate in region A can be written

\begin{equation}
\label{chap:analysis:equation:dy_est1}
N_{\mathrm{DY},i}^{A,est.} = \frac{N_{\mathrm{DY},\mathrm{data}}^{C}}{N_{\mathrm{DY},\mathrm{data}}^{D}}\cdot{N_{\mathrm{DY},\mathrm{data},i}^{B}},
\end{equation}

\noindent
where $N_{\mathrm{DY},\mathrm{data}}$ is the number of data events in the control
region with the non-$\mathrm{DY}$ component, predicted by MC,
subtracted. The index $i$ labels the BDT bin. The \etmiss efficiency
is not binned in BDT-- i.e. the same ratio is applied for each bin. As
for the top estimate, due to limited statistics in the high BDT bins,
the control regions for these bins are merged. Therefore, the relative
rates in these two bins are taken from the MC prediction. The
resulting NFs, defined as
$N_{\mathrm{DY},i}^{A,est.}/N_{\mathrm{DY,i}}^{A,MC}$, are $1.0 \pm
0.2$ in the lowest BDT bin, and $0.9 \pm 0.3$ in the two highest bins.

\begin{table}
\centering
%\vspace{0.5cm}
{\footnotesize
    \centering
    \begin{tabular}{|c||c|}
        \hline
        & \\
        \textbf{Region A (SR)}                          &
        \textbf{Region C }                           \\
        & \\
        $\calomet > 45 \GeV$                            & $\calomet >
        45 \GeV$                          \\
        $\mll < 75 \GeV$                                & $|\mll -
        m_Z| < 15 \GeV$              \\
        & \\
        \hline
        & \\
        \textbf{Region B}             & \textbf{Region D} \\
        & \\
        $25 \GeV < \calomet < 45 \GeV$          & $25 \GeV < \calomet
        < 45 \GeV$                        \\
        $\mll < 75 \GeV$                                & $|\mll -
        m_Z| < 15 \GeV$                      \\
        & \\
        \hline
    \end{tabular} }
\caption[Summary of the regions used for the data-driven \ZDY~estimate.]{Summary of the regions used for the data-driven
  \ZDY~estimate (ABCD method). Region A corresponds to the signal
  region. The BDT template is taken from data in region B, and the
  efficiency of the \calomet cut is determined in the $Z$ peak
  (regions C and D).}
\label{chap:analysis:tab:bdt_abcd_cartoon}
\end{table}

This so called ``ABCD method'' is predicated on two assumptions. The
first is that the BDT response is not correlated to calorimeter
\etmiss, allowing the BDT shape template to be taken from the low
\etmiss CR. The second implicit assumption is that the calorimeter
\etmiss cut efficiency is not correlated with \mll. This assumption
is needed to apply the low to high \etmiss extrapolation factor from
the $Z$ peak in the low \mll region. The validity of these two
assumptions is empirically tested in MC, and any breakdown is
accounted for in the assignment of systematics uncertainties. 

The absence of a correlation between the BDT response and calorimeter \etmiss
is due to the fact that jet-corrected \trkmet is used in the only two
BDT inputs that depend on \etmiss, \mT and \pTtot. The linear
correlation coefficients between the BDT inputs and both calorimeter
and track \etmiss are shown
in~\ref{chap:analysis:tab:bdt_met_corr}. With the exception of \mT,
the correlation between calorimeter \etmiss and the BDT inputs is less
than 0.1. To account for any correlation, the difference in the shape
of the BDT template in regions A and B is taken as an
uncertainty. This difference is computed in A{\sc lpgen}+H{\sc
  erwig} and A{\sc lpgen}+P{\sc ythia}, and the largest uncertainty
between the two MC generators is assigned. The template comparisons
are shown in figure~\ref{chap:analysis:fig:bdt_met_corr}, with
uncertainties corresponding to 4\%, 10\%, and 60\% in bins of BDT. 

\begin{table}[h]
\centering
\renewcommand{\arraystretch}{1.2}
%{\small
%\resizebox{0.8\textwidth}{!}
{
\begin{tabular}{ l | c | c }
\hline
BDT input & \calomet & \trkmet \\
\hline
\mT & 0.14 & 0.31 \\
\pTtot & 0.00 & 0.07 \\
\mjj & 0.08 & 0.11 \\
\SumMlj & 0.08 & 0.10 \\
\dphill & -0.04 & 0.02 \\
\mll & 0.03 & 0.04 \\
\lepEtaCent & 0.02 & 0.00 \\
\dyjj & -0.02 & 0.05 \\
\hline
\end{tabular}
}
\caption[Linear correlation coefficients between BDT inputs and
  \etmiss quantities for \ZDY.]{Linear correlation coefficients between BDT
  inputs and \etmiss quantities for \ZDY.}
\label{chap:analysis:tab:bdt_met_corr}
\end{table}

\begin{figure}[h]
  \centering
  \includegraphics[width=0.45\textwidth]{fig/analysis/bdt_met_corr_herwig.eps}
  \includegraphics[width=0.45\textwidth]{fig/analysis/bdt_met_corr_pythia.eps}
   \caption[]{Comparison of BDT template for \ZDY in
     $25 \gev < \etmiss< 45 \gev$~region (blue) and $\etmiss >
     45 \gev$~region (red). The baseline \ZDY sample A{\sc lpgen}+H{\sc
       erwig} is shown on the left and A{\sc lpgen}+P{\sc ythia} is on
     the right. The difference in template between low \etmiss region and
  high \etmiss region is assigned as an uncertainty on \ZDY.}
  \label{chap:analysis:fig:bdt_met_corr}
\end{figure}

The second assumption, that the calorimeter \etmiss efficiency is not
correlated to \mll, is true for the leading order $Z$+$2j$
processes. However, for higher order processes with ISR, the
lepton-jet system recoils against the QCD radiation, thereby
increasing \mll. The soft hadronic activity associated with ISR is
more susceptible to mis-measurement, translating to a larger calorimeter
\etmiss resolution. With larger \etmiss tails and higher \mll in ISR
events, a correlation between these two quantities is
induced. Therefore, the efficiency of the \calomet$>45$~\gev~cut in the $Z$ peak is expected
to be greater than that in the \mll$<75$~\gev~ region. The degree to
which the efficiencies disagree is referred to as the non-closure of
the method, and is quantified by

\begin{equation}
f_{\textrm{non-closure}} = \frac{N_A/N_B}{N_C/N_D}
\label{chap:analysis:equation:abcd_nonclosure}
\end{equation}

\noindent
where the event yields are taken from \ZDY MC. The \ZDY estimate in
the SR, given by equation~\ref{chap:analysis:equation:dy_est1}, is
scaled by this factor $f_{\textrm{non-closure}}$ to account for the
\etmiss efficiency difference. Moreover, the difference between
$f_{\textrm{non-closure}}$ and unity (17\%) is assigned as an
uncertainty. (discuss study of non-closure in data?). The \etmiss
efficiency in the $Z$ peak ($N_C/N_D$), as measured in data, is $0.43
\pm 0.03$, which is constituent with the value from \ZDY MC of $0.47
\pm 0.04$. The statistical uncertainty on this efficiency is taken as
an uncertainty.

The $Z$ boson processes in which the $Z$ decays to $\tau\tau$
contribute in both the \emme and \eemm channels. The normalization of
this background is taken from a control region that includes all of
the preselection cuts with the exception of the
$Z\rightarrow{\tau\tau}$ veto. Instead, $Z\rightarrow{\tau\tau}$ is
enhanced by requiring that $|\mtt - m_Z| < 25 \gev$. Additionally, the
cut $\mll < 80 \gev$~is applied to \eemm events, and to pick out a
more signal-like phase space region, $\textrm{BDT} > -0.48$ is
required. In order to increase the statistics in this region, thereby
decreasing the statistical uncertainty on the NF, the \emme and \eemm
channels are merged for a common NF. The resulting NF is $1.2 \pm
0.3$. The same normalization is applied across the BDT spectrum due to
limited $Z\rightarrow{\tau\tau}$ statistics at high BDT. 
