
Top background is comprised mainly of \ttbar~(Figure), with $s$ and $t$
channel single top and $Wt$ contributions as well (denoted ST). The parton
level process for \ttbar~is simulated in
\POWHEG at NLO in QCD, while \PYTHIAns 6 is used for parton showering and
hadronization. The LO PDF set CTEQ6L1 is used with \PERUGIA 2011 as
the underlying event tune. Past iterations of the VBF analysis in this
decay channel used \MCATNLO to model
\ttbar~\cite{bib:hww_moriond_2013}, but \POWHEGns +\PYTHIA models jet
kinematic distributions better in the VBF phase space region. However, leptons generated in
\MCATNLO model the data better than \POWHEG. To account for these
modelling differences, theoretical uncertainties are assigned to the
predicted \ttbar~yield in the signal region. For ST,
$s$ channel and $Wt$ use \POWHEGns +\PYTHIAns 6 as well, while
\textsc{AcerMC}+\PYTHIAns 6 is used to model $t$-channel ST. The same PDF
and UE tune is used for all top processes. 

The \ttbar~normalization is scaled to the cross section for $pp$
collisions at \sqrts$=8$ \tev~for a top quark mass of
$172.5 \gev/c^2$, $\sigma_{t\bar{t}}=252.9^{+15.3}_{-16.3}$~pb. This value has been
calculated at NNLO in QCD including a resummation of
next-to-next-to-leading logarithmic (NNLL) soft gluon
terms~\cite{bib:Cacciari:2011hy,bib:Beneke:2011mq,bib:Baernreuther:2012ws,bib:Czakon:2012zr,bib:Czakon:2012pz,bib:Czakon:2013goa,bib:Czakon:2011xx}.
The total cross section for \ttbar~has been measured to be
$237.7\pm11.3$~pb in the leptonic decay channel with one
electron and one muon in the final
state~\cite{bib:ttbar_cross_section}, agreeing with the
theory calculation within the QCD scale uncertainties. The $s$ channel
ST, $t$ channel ST, and $Wt$ normalizations are also scaled to the
NNLO and NNLL cross sections of $5.61\pm0.22$~pb~\cite{bib:Kidonakis:2010tc},
$87.76^{+3.44}_{-1.91}$~pb~\cite{bib:Kidonakis:2011wy}, and
$22.37\pm1.52$~pb~\cite{bib:Kidonakis:2010ux}, respectively. These
computations are compatible with the respective measurements in
ATLAS~\cite{bib:tchan_cross_section,bib:Wt_cross_section}. 

In spite of the high level of agreement observed between theory
calculations and ATLAS top measurements, the top normalization is
constrained using a top-rich control region (CR). The cross section
measurements in
ATLAS,~\cite{bib:ttbar_cross_section,bib:Wt_cross_section}, require
the selected jets to be central ($|\eta|<2.5$), whereas in the VBF
analysis, due to the signal topology, the $\eta$ requirement is looser,
$|\eta|<4.5$. Dedicated top measurements have yet to probe this region
to test existing theoretical models, motivating the use of a
data-driven top estimate. Because the kinematic
shapes are similar for \ttbar~and ST, these processes share a common
normalization. All of the selection cuts applied in the SR are also
applied in the top CR, with the exception of the BJV. Instead of
requiring zero $b$-tags, exactly one $b$-tag is required, enriching
this region with top background while while minimizing contamination from
sources without heavy flavor quarks in the final state. The top CR is
subdivided into BDT bins, and the top
normalization is computed separately in each bin. This approach is
used because the BDT spans a large and relatively unknown region of
phase space, and there is no \textit{a priori} reason to assume that the
normalization is constant across the BDT spectrum. The normalization
factor in BDT bin $i$ can be expressed as

\begin{equation}
\label{chap:analysis:equation:top_nf}
NF_i = \frac{N_{data}^{CR} - N_{non-top}^{CR}}{N_{top}}
\end{equation}


