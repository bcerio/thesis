
The analysis is performed on $pp$ collision data recorded by the ATLAS
detector (chapter~\ref{chap:lhc_atlas}). Data are partitioned in
time into runs and if the detector subsystems are functioning
sufficiently well in a given run (should I go into more detail?),
the data for that run are included. This thesis focuses on
20.3~fb$^{-1}$ of data collected in 2012 at $\sqrts =8 \tev$. For the
final statistical results, a smaller dataset of
4.5~fb$^{-1}$ at $\sqrts =7 \tev$ is included
(section~\ref{chap:analysis:sec:reanalysis}).

Monte Carlo simulations are relied on for the prediction of signal and
background observables. The hard scattering process, subsequent
showering and hadronization of final state quarks and gluons, and the
response of the detector are all simulated with MC programs. The
absolute predictions are then obtained by scaling the MC distributions
by the factor 

\begin{equation}
\textrm{norm} = \mathcal{L}\cdot{\sigma}/N_{\mathrm{total}},
\label{chap:analysis:equation:mc_norm}
\end{equation}

\noindent
where $\mathcal{L}$ is the integrated luminosity, $\sigma$ is the best
available cross section calculation for the process, and
$N_{\mathrm{total}}$ is the total number of events generated, or in
the case of weighted MC, the sum of
weights. Table~\ref{chap:analysis:tab:mc_summary} summarizes the MC
programs used for each signal and background process, as well as the
cross section to which the prediction is scaled. Each cross section
includes the relevant branching fraction for the $\ell\nu\ell\nu$
final state integrated over all lepton combinations. 

\begin{table}[h]
  \centering
%{\small
\resizebox{0.8\textwidth}{!}{
  \begin{tabular}{llrr}
    \hline
    Process & Generator & \hspace*{-3mm}$\sigma\cdot\mathrm{Br}$($8 \tev$)
  (pb) \\%\hspace*{-3mm}$\sigma\cdot\mathrm{Br}$(7TeV) (pb)\\
    \hline
    VBF $H\rightarrow WW$  & \POWHEG~+~\PYTHIAns8 & $36\cdot 10^{-3}$ \\%& $28\cdot 10^{-3}$ \\
    ggF $H\rightarrow WW $  & \POWHEG~+~\PYTHIAns8 & 0.435\\% &  0.341 \\
    $WH/ZH$ $H\rightarrow WW $ & \PYTHIAns8 (\PYTHIAns6) & $25\cdot 10^{-3}$ \\% & $21\cdot 10^{-3}$ \\
    \hline
    $\ttbar$ dileptonic & \POWHEG~+\PYTHIAns6 & 26.6 \\%& 18.6\\
    % single top NB: sum s-ch, t-ch, Wt
    $tW/tb$ leptonic      & \POWHEG~+~\PYTHIAns6 & 4.17\\% & 3.15
    $tqb$ leptonic        & \textsc{AcerMC}~+~\PYTHIAns6 & 28.4\\% & 20.7\\         % checked
    QCD $WW + 2$ jets & \SHERPA & 0.568\\% & - \\
    EW $WW + 2$ jets & \SHERPA & 0.039\\% & 0.027\\
    $gg\rightarrow WW$ & \GGTOWW~+~\HERWIG & 0.20\\% & 0.14\\
    % W+jets NB: BR to single lepton flavour multiplied by 3 flavours
    inclusive $W$ & \ALPGEN~+~\HERWIG & $37\cdot 10^{3}$\\% & $31\cdot 10^{3}$\\
    % Z/gamma*+jets NB: BR to single lepton flavour multiplied by 3 flavours
    inclusive $Z/\gamma^{\star} (m_{ll} \ge 10 GeV)$ & \ALPGEN~+~\HERWIG & $16.5\cdot 10^{3}$\\% & $14.9\cdot 10^{3}$\\
    EW $Z/\gamma^{\star}$ & \SHERPA & 5.36 (inc. t-ch)\\% & 2.26\\
    $W(Z/\gamma^{\ast}) $     & \POWHEG~+~\PYTHIAns8 & 12.7\\% & 10.8\\
    $W(Z/\gamma^{\ast}) (m_{(Z/\gamma^{\ast})} < 7~\GeV)\ $ & \SHERPA & 12.2\\% & 10.6 \\ % checked
    $Z^{(\ast)}Z^{(\ast)} \to 4l(2l2\nu)$ & \POWHEG~+~\PYTHIAns8 & 0.73(0.50)\\% & 0.64(0.42)\\
    EW $WZ + 2$ jets  & \SHERPA & $13\cdot 10^{-3}$\\% & $8.5\cdot 10^{-3}$\\
    EW $ZZ + 2$ jets $(4l,ll\nu\nu)$ & \SHERPA & $73\cdot 10^{-5}(12\cdot 10^{-4})$\\% & $53\cdot 10^{-5}(8.8 \cdot 10^{-4})$\\
    $W\gamma$ & \ALPGEN~+~\HERWIG & 369\\% & 313\\
    $Z\gamma$($p_{T}^{\gamma} > 7GeV$) & \SHERPA & 163\\% & - \\
    \hline
  \end{tabular}
}
\caption[MC sample summary.]{Signal and background process summary
    with MC generators used and process $\sigma\cdot\textrm{Br}$,
  where the branching ratio is integrated over all flavor channels
  resulting in two charged leptons.}
  \label{chap:analysis:tab:mc_summary}
\end{table}

The signal process, VBF Higgs production in the \ww decay channel, is
modeled with \POWHEG~\cite{bib:Nason:2009ai} interfaced
to \PYTHIAns8~\cite{bib:Sjostrand:2007gs} for the parton shower. The
cross section is computed at NLO in QCD and
EW~\cite{Ciccolini:2007jr,Ciccolini:2007ec,Arnold:2008rz}, with
approximate NNLO QCD corrections~\cite{Bolzoni:2010xr}. The branching
fraction for the \ww decay for this VBF process and the other Higgs processes in
this analysis is derived from the \hdecay
program~\cite{Djouadi:1997yw}. The final $\sigma\cdot{\mathrm{Br}}$~is
0.036~pb at $m_H = 125$~\gev. Because $VH$ Higgs production probes the
same Higgs coupling as VBF, these processes are lumped into
signal. The cross sections for $ZH/WH$ are calculated at NNLO in
QCD~\cite{Han:1991ia,Brein:2003wg} with NLO EW radiative
corrections~\cite{Ciccolini:2003jy}. At $m_H = 125 \gev$ the resulting
cross sections are $\sigma_{ZH} = 9.4$~fb and $\sigma_{ZH} = 15.9$~fb
for the processes in which the $W$ bosons from the Higgs decay
leptonically. 

The ggF$+2$j process is also simulated with \POWHEG~\cite{bib:Alioli:2008tz} interfaced with
\PYTHIAns8. The cross section is computed at NNLO in
QCD~\cite{Djouadi:1991tka,Dawson:1990zj,Spira:1995rr,Harlander:2002wh,Anastasiou:2002yz,Ravindran:2003um},
with NLO electroweak corrections~\cite{Aglietti:2004nj,Actis:2008ug}
and soft gluon resummations up to
next-to-next-to-leading-log (NNLL)~\cite{Catani:2003zt}. Due
to known deficiencies in the Higgs $p_T$ spectrum in \POWHEG, the
Higgs $p_{\mathrm{T}}$ is re-weighted to the NLO + NNLL prediction
from \HqT~\cite{deFlorian:2011xf}. The $\sigma\cdot{\mathrm{Br}}$ is
0.435~pb, a factor of 12 larger than VBF. 

Top background is comprised mainly of \ttbar, with smaller
contributions from $s$ and $t$ channel single top and $Wt$ (labeled
collectively as ST). The
hard scatter for \ttbar~is simulated in \POWHEG~\cite{bib:Frixione:2003ei} at NLO in QCD,
while \PYTHIAns6~\cite{bib:Sjostrand:2006za} is used for parton showering and
hadronization. The LO PDF set CTEQ6L1 is used with \PERUGIA~2011 as
the underlying event tune. For ST,
$s$ channel and $Wt$ use \POWHEGns~\cite{bib:Alioli:2009je,bib:Re:2010bp}~+~\PYTHIAns6 as well, while
\textsc{AcerMC}~\cite{bib:Kersevan:2004yg}~+~\PYTHIAns6 is used to model $t$-channel ST. The same
PDF and UE tune is used for all top processes. The
\ttbar~normalization is scaled to the cross section for $pp$
collisions at $\sqrts = 8 \tev$ for a top quark mass of
$172.5 \gev/c^2$, $\sigma_{t\bar{t}}=252.9^{+15.3}_{-16.3}$~pb. This
value has been calculated at NNLO in QCD including a resummation of
NNLL soft gluon
terms~\cite{bib:Cacciari:2011hy,bib:Beneke:2011mq,bib:Baernreuther:2012ws,bib:Czakon:2012zr,bib:Czakon:2012pz,bib:Czakon:2013goa,bib:Czakon:2011xx}.
The total cross section for \ttbar~has been measured in ATLAS to be
$237.7\pm11.3$~pb in the leptonic decay channel with one
electron and one muon in the final
state~\cite{bib:ttbar_cross_section}, agreeing with the
theory calculation within the QCD scale uncertainties. The $s$ channel
ST, $t$ channel ST, and $Wt$ normalizations are also scaled to the
NNLO and NNLL cross sections of
$5.61\pm0.22$~pb~\cite{bib:Kidonakis:2010tc},
$87.76^{+3.44}_{-1.91}$~pb~\cite{bib:Kidonakis:2011wy}, and
$22.37\pm1.52$~pb~\cite{bib:Kidonakis:2010ux}, respectively. These
computations are compatible with the respective measurements in
ATLAS~\cite{bib:tchan_cross_section,bib:Wt_cross_section}.

Standard model production of a pair of $W$ bosons is split into two
classes, depending on how the final state quarks or gluons are
produced. ``QCD \ww'' jets come from QCD vertices, while ``EW \ww''
jets are produced via electroweak couplings. The QCD \ww prediction is from \SHERPA
1.4.1~\cite{bib:Gleisberg:2008ta}, which is used to
simulate the hard-scatter, parton shower, and hadronization for all
$q\bar{q}/qg/\bar{q}g\rightarrow{WW}$ diagrams. Events are generated
at LO in QCD with up to three jets at matrix element level, and the
$W$ bosons are forced to decay leptonically. The overall normalization is
scaled to the total \ww cross section at $\sqrts=8 \tev$, as computed
in \MCFM~\cite{bib:Campbell:2011bn}: $\sigma_{tot}(\wwlnln) = 5.679$~pb. Given that the branching
fraction of $W\rightarrow{\ell\nu}$ is 0.1082 and there are nine
lepton combinations, the corresponding total cross section is
$\sigma_{tot}(\ww) = 53.90$~pb. The total \ww cross section has been
measured in ATLAS in the \ww leptonic decay channels and the exclusive
zero jet bin~\cite{bib:ww_cross_section}. The result, which includes
resonant \ww production from $H$, is
$\sigma_{tot}(WW) = 71.4^{+1.2}_{-1.2} \textrm{(stat)} ^{+5.0}_{-4.4}
\textrm{(syst)}^{+2.2}_{-2.1}\textrm{(lumi)}$~pb, significantly larger
than the \MCFM computation which includes $H$, $\sigma_{MCFM}(WW) =
58.7^{+3.0}_{-2.7}$~pb. However, the ATLAS measurement is done in a
phase space region that is orthogonal to this analysis, and in fact,
the \MCFM calculation does not include diagrams with two partons in the final
state. The normalization in the 2j bin relies on the LO jet
multiplicity prediction from \SHERPA. For the production of
$gg\rightarrow{WW}$ through a quark loop, the NLO generator \GGTOWW~\cite{bib:Binoth:2006mf} is
used with the parton showering program
\HERWIG~\cite{bib:Corcella:2000bw}, though this contribution is
insignificant compared to the other $WW$ processes. 

The EW \ww processes are also simulated with \SHERPA, but with exactly two
jets in the hard scatter. The LO cross section for EW \ww with the
$W$ bosons decaying to leptons, computed in \SHERPA, is $\sigma(\wwlnln) =
39.68$~fb. Events with Higgs couplings are not simulated; instead, the
interference between Higgs and \ww processes is assessed as a
uncertainty on the cross section. 

Like $WW$, SM \zjets processes, also referred to as Drell Yan or
$Z/\gamma^\ast$, are split into QCD and EW categories
depending on the nature of the jets. QCD \zjets is simulated with the
LO event generator \ALPGEN~\cite{bib:Mangano:2002ea} interfaced to
\HERWIG. To enhance statistics, events are generated with a dilepton
filter, requiring $\mll>10 \gev$, and a filter to select events with jets that are VBF-like (at
least two jets with $\pt>15 \gev$, $|\eta| < 5.0$, $\mjj>200 \gev$,
and $\dyjj>2.0$). VBF-filtered samples are merged with unfiltered
samples for full phase space coverage. To enhance statistics for events with a
high \pt~photon ($\pt>7 \gev$) in addition to the $Z$ boson, a
dedicated \SHERPA sample is used, and the phase space overlap is removed at truth
level. EW \zjets processes are modeled with \SHERPA, with a generator
filter to select dilepton events with $\mll>7 \gev$. 

Non-$WW$ diboson backgrounds are sub-dominant contributions in the VBF
analysis. For $WZ/W\gamma^{\ast}$, events are generated in \POWHEG~\cite{bib:Melia:2011tj},
which treats interference between $Z$ and $\gamma^{\ast}$ diagrams
properly. Because \POWHEG can not produce events at low dilepton mass,
the phase space in which $m_Z/\gamma^{\ast} < 7 \gev$ is
modeled with \SHERPA with up to one final state parton computed at the
matrix element level. This prediction is then scaled to the NLO cross
section computed in \MCFM. $ZZ$ processes are also modeled with
\POWHEG. For the EW processes without QCD vertices at LO, \SHERPA is
used for $WZ$, $W\gamma^{\ast}$, and $ZZ$. 

The stable particles produced after the generation of the hard scatter
and hadronization are then propagated through a full simulation of the
ATLAS detector implemented in \GEANT. (Finish section after writing
simulation section). 





