
The primary aim of the analysis presented in this thesis is to measure the rate at which the
Higgs boson is produced via vector boson fusion. Previous Higgs
measurements, including the ones in which the Higgs was discovered,
have lumped together ggF, VBF, and VH into a single signal controlled
by the the same strength. This analysis, by contrast, considers only the
Higgs processes with two $VVH$ couplings, VBF and $VH$, to be
signal, with the assumption that SM ggF production at
$m_H=125$~\gev~has already been discovered. Therefore, the
optimization of the analysis was done at $m_H=125$~\gev~, with ggF $H$
being produced at the SM rate included in the background-only hypothesis. In
addition to measuring the rate of VBF production, the probability of
observing the data assuming the null hypothesis to be true is
extracted. In the presence of an excess in the observed data with
respect to the background-only hypothesis, this
corresponds to the statistical significance of the excess.

Dedicated VBF rate measurements have been done in ATLAS in the \wwlnln
channel with 25~fb$^{-1}$ of data, with an observed significance of 2.5$\sigma$ and
a measured signal strength of $1.66\pm 0.79$ times the
$\sigma_{\mathrm{VBF}}\cdot{\mathrm{Br}}$ predicted by the
SM~\cite{bib:hww_moriond_2013}. The analysis presented in this thesis
improves on the previous analysis in
many respects, most notably the use of a multivariate technique called
a boosted decision tree (BDT) which is explained in chapter
XX. This chapter details the BDT-based VBF analysis, describing in
detail the selection of physics objects and events
(sections~\ref{chap:analysis:sec:objects} and~\ref{chap:analysis:sec:event_selection}), the
BDT inputs and and validation
(sections~\ref{chap:analysis:sec:bdt_inputs} and~\ref{chap:analysis:sec:bdt_validation}),
data-driven background estimation
methods (section~\ref{chap:analysis:sec:dd_backgrounds}), systematic
uncertainties (section~\ref{chap:analysis:sec:systematics}),
and finally the results (section~\ref{chap:analysis:sec:results}). A
more in depth discussion of the statistical methods and results can be
found in the following chapter. This chapter will conclude with a
summary of the analysis of a smaller dataset collected at $\sqrts =
7 \tev$ in 2011, as well as a brief description of the \hww analyses
in other jet bins
