
Due to limited resources, it is not possible to store the information
associated with each collision event at the design collision rate of
1~GHz. The trigger system selects potentially interesting events,
thereby reducing the effective collision rate, before events are
stored for downstream analysis. It is organized into three levels---
Level-1 (L1), Level-2 (L2), and event filter (EF). This three level
structure seeks to deal with a
fundamental problem of selecting potentially interesting events,
namely that such a decision requires partial event reconstruction,
which by its nature, requires time and memory. The levels progress
from very fast and coarse reconstruction applied to events which are
collected at a high rate to a more refined reconstruction
applied to a small subset of these events. 

The L1 trigger integrates a limited amount of information from all of
the calorimeter
subsystems and the muon trigger chambers to identify high
\et~electrons, photons, jets, and muons. The calorimeter trigger
system searches for energy deposits in a coarse granularity cell of
$\Delta\eta \times \Delta\phi = 0.1 \times 0.1$ that fall above a
series of configured thresholds. If the multiplicity of these deposits
falls above an energy-threshold-dependent cut-off value, then the
event passes the trigger. The muon trigger system uses RPC (TGC)
measurements in the barrel (end-caps) to identify patterns that are
consistent with a muon originating from the collision point. Starting
with a muon hit in the second chamber layer, the trigger searches for
additional hits along a path formed by extrapolating to the collision
point, and if a minimum number of hits is found, the pattern is
considered a muon candidate. If the number of muon candidates for the
same bunch crossing falls above a threshold, the L1 muon trigger is
accepted. The trigger is binned into 6 transverse momentum bins. The
L1 trigger reduces the event rate from 1~GHz to 75~kHz with each
trigger decision executed in less than $2.5 \mu$s. L1 trigger
processing is done in the front-end electronics system on the
detector, as is the storage of event data in buffers. Once the L1
trigger is passed, the data are sent away from the detector to readout
drivers, awaiting the L2 trigger. 

The L2 trigger is considered a high level trigger (HLT) in that it makes a
trigger decision based on the full granularity of the detector and
even some information from the ID. As opposed to the L1 triggers that
use front-end hardware to process an event, the L2 trigger system is
built around a specialized software-based framework that runs on a
computer farm. L2 triggers consider small event data fragments
associated with regions of interest (ROIs) that are defined by the L1
trigger. Data for each ROI is typically on the order of 1\% of the
total data in the event, which speeds up the L2 trigger processing and
minimizes the amount of data transferred to the farm. The L2 trigger
algorithms iteratively pull data from the readout drivers and determine
whether an ROI satisfies the hypothesis for a given particle. If the
algorithm determines that the ROI is not consistent with a particle,
the next ROI data is pulled and the process repeats. If none of the
ROIs are found to be consistent with particles, the event is
rejected. In its current form, the L2 system can only accomodate an
event rate of 40~kHz, about half of the design trigger rate for the L1
trigger. L2 further reduces the trigger rate to 3.5~kHz, with a
processing time of around 40~ms for each event. 

The final trigger level is the event filter (EF) trigger. It is also a
HLT, but instead of using full granularity portions of the event (ROIs), it
improves on the L2 trigger by incorporating all of the event data into
the processing algorithms. Event processing is done offline in
computer clusters at an average rate of four seconds
per event, and the resulting trigger rate is reduced to 200~Hz. Events
that pass the EF trigger are transferred to the CERN computer center
for permanent storage. The raw data for each event amounts to around
1.3~MB on average. 
