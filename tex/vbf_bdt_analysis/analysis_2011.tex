
The BDT analysis has been optimized for the $20.3~\ifb$ dataset
collected by the ATLAS detector
at $\sqrts = 8 \tev$. In addition to this dataset, a smaller dataset
of $4.5~\ifb$ integrated at $\sqrts = 7 \tev$ has been analyzed. Since
this dataset is substantially smaller, expected gains in sensitivity
are expected to be smaller, and therefore an independent optimization
has not been performed for this analysis. Instead, the BDT developed
for the $8 \tev$ dataset is applied in the $7 \tev$ analysis, under
the assumption that the difference in $\sqrt{s}$ and pile-up conditions do
not have a significant impact on the kinematics of the final state
objects. Another compelling consideration motivating the use of the
same BDT is that because the BDT defines the signal region of phase
space, and the $8 \tev$ theory uncertainties have been computed for
this region, the theory uncertainties can be recycled for the $7 \tev$
analysis. In the following section, the $\sqrts = 7 \tev$ BDT analysis
is briefly documented. 

\subsection{Backgrounds}
\label{chap:analysis:sec:reanalysis:subsec:back}

Backgrounds for the $7 \tev$ analysis are predicted with MC
simulation and, when necessary, data-driven estimates. The MC
generators are identical to those of the $8 \tev$ analysis, with the
exception of the QCD $WW$ generator, for which \POWHEG interfaced to
\PYTHIA 6 is used, due to the unavailability of \SHERPA. Since \POWHEG
is only able to generate inclusive $WW$ at NLO, the second
high-\pt~jet is produced in the parton shower, leading to potential
mis-modeling of di-jet kinematic variables. Due to the absence of a
sufficiently pure \wwtwoj control region, the modelling of this
background can not be validated. Instead, large theory systematics are
assigned. The MC predictions are normalized to their respective cross
sections at $\sqrts = 7 \tev$, shown in
table~\ref{chap:analysis:tab:mc_summary_2011}, which are generally on
the order of 20\% smaller than those at $8 \tev$. The VBF cross
section decreases by 22\%, which is compensated for by smaller
background cross sections. 

\begin{table}[h]
\centering
\resizebox{0.8\textwidth}{!}{
\begin{tabular}{l|r|r}
\hline
Process & \hspace*{-3mm}$\sigma\cdot\mathrm{Br}$(7 \tev)
(pb) & change w.r.t. $8 \tev$\\
\hline
VBF $H\rightarrow WW$ & $28\cdot 10^{-3}$ & -22\%\\
ggF $H\rightarrow WW $ & 0.341 & -22\%\\
$WH/ZH$ $H\rightarrow WW $ & $21\cdot
10^{-3}$ & -16\%\\
\hline
$\ttbar$ dileptonic & 18.6 & -30\%\\
$tW/tb$ leptonic & 3.15 & -24\%\\
$tqb$ leptonic   & 20.7 & -27\%\\
QCD $WW + 2$ jets \textdagger & 0.468 & -18\%\\
EW $WW + 2$ jets & 0.027 & -31\%\\
$gg\rightarrow WW$ & 0.14 & -30\%\\
inclusive $W$ & $31\cdot 10^{3}$ & -16\%\\
inclusive $Z/\gamma^{\star} (m_{ll} \ge 10 GeV)$ & $14.9\cdot 10^{3}$ & -10\%\\
EW $Z/\gamma^{\star}$ & 2.26 & n/a \\ 
$W(Z/\gamma^{\ast}) $ & 10.8 & -15\%\\
$W(Z/\gamma^{\ast}) (m_{(Z/\gamma^{\ast})} < 7~\GeV)\ $ & 10.6 & -13\%\\
$Z^{(\ast)}Z^{(\ast)} \to 4l(2l2\nu)$ & 0.64(0.42) & -12\%(-16\%)\\
EW $WZ + 2$ jets & $8.5\cdot 10^{-3}$ & -35\%\\
EW $ZZ + 2$ jets $(4l,ll\nu\nu)$ & $53\cdot
10^{-5}(8.8\cdot 10^{-4})$ & -27\%(-27\%)\\
$W\gamma$ & 313 & -15\%\\
\hline
\end{tabular}
}
\caption[Cross section summary for $\sqrts = 7 \tev$ analysis.]{Cross
  sections for $\sqrts = 7 \tev$ MC predictions and associated change
  with respect to
  $\sqrts = 8 \tev$. MC generators are the same as those in the
  $8 \tev$ analysis, shown in
  table~\ref{chap:analysis:tab:mc_summary}, except for QCD \ww \textdagger.}
\label{chap:analysis:tab:mc_summary_2011}
\end{table}

Due to limited data statistics in CRs and
limited MC simulation statistics in the SRs, the binning of the BDT
response is coarser than that of the $8 \tev$ analysis. However, in
order to use the BDT-dependent theory
systematics computed at $8 \tev$, the same bin boundaries are used and
low statistics bins are merged. In
the \emme channel, the two high BDT bins in the $8 \tev$ analysis are
merged, resulting in two bins with boundaries [-0.48,0.3,1.0]. In the
\eemm channel, the fit is performed on a single BDT bin [-0.48,1.0].

Data-driven background estimation techniques are inherited from the
$8 \tev$ analysis. The top background is normalized from the same CR,
with the exception of the BDT binning. Due to limited stats, the NF is
computed with a control region with all BDT bins merged and is applied across
the BDT spectrum. The resulting purity is 91\% and the NF is $0.8 \pm 0.3$. \ZDYll is
estimated with the ABCD method for a single BDT bin, resulting in an
estimate of $N_{\mathrm{DY}}^{A,est.} = 2.3 \pm 1.0$, which is
compatible with the prediction from MC simulation of $1.4 \pm 0.5$
events. 

\begin{table}[p!]
\begin{center}
\renewcommand{\arraystretch}{1.2}
\resizebox{0.9\textwidth}{!}{
\begin{tabular}{l || c c | c c c c c c }
\hline
Uncertainty & \multirow{2}{*}{Signal} & Total &
\multirow{2}{*}{top} & \multirow{2}{*}{$WW$} & \multirow{2}{*}{ggF} &
Non-$WW$ & \multirow{2}{*}{\ZDY} & \multirow{2}{*}{Fakes} \\
Source & & Back & & & & Diboson & & \\
\hline
$\textrm{Trigger}$ & - & 0.2 & 0.5 & - & - & 0.4 & - & - \\
$\textrm{electron SF}$ & 1.5 & 1.1 & 1.2 & 1.0 & 1.4 & 1.3 & 1.4 & -
\\
$\textrm{electron reso.}$ & 0.1 & 0.4 & 1.0 & 0.1 & 0.4 & 2.9 & 1.3 &
- \\
$\textrm{electron scale}$ & - & 0.2 & 0.7 & 0.5 & - & 3.5 & 1.3 & - \\
$\textrm{electron iso.}$ & 2.0 & 2.0 & 2.1 & 2.1 & 2.0 & 2.0 & 2.1 & -
\\
$\textrm{muon SF}$ & - & - & - & - & - & - & - & - \\
$\textrm{muon reso.}$ & 0.1 & 0.7 & 1.6 & 0.1 & - & - & 0.1 & - \\
$\textrm{muon scale}$ & - & 0.5 & 1.3 & - & - & - & 0.1 & - \\
$\textrm{muon iso.}$ & 1.0 & 1.0 & 1.0 & 1.0 & 1.0 & 1.0 & 1.0 & - \\
$\textrm{JES detector}$ & 0.1 & 0.7 & 0.1 & 3.7 & 0.9 & 1.2 & 1.1 & -
\\
$\textrm{JES}~\eta~\textrm{stat.}$ & - & 0.1 & - & 2.0 & - & - & 1.3 &
- \\
$\textrm{JES modelling}$ & 0.7 & 0.5 & 3.9 & 3.0 & 0.3 & 0.6 & 7.4 & -
\\
$\textrm{JES Stat.}$ & 0.1 & 1.0 & - & 1.4 & 0.3 & 0.3 & 6.1 & - \\
$\textrm{JES}~b~\textrm{jet}$ & - & 0.6 & 1.6 & 0.1 & - & 0.2 & 0.1 &
- \\
$\textrm{JES}~\eta~\textrm{model}$ & 3.5 & 7.5 & 13.9 & 6.2 & 7.1 &
0.8 & 0.6 & - \\
$\textrm{JES flavor}$ & 1.6 & 2.5 & 2.1 & 0.4 & 4.8 & 11.6 & 1.3 & -
\\
$\textrm{JES flav. response}$ & 0.4 & 2.2 & 3.2 & 2.7 & 0.9 & 0.4 &
1.4 & - \\
$\textrm{JES}~\left \langle \mu \right \rangle$ & 1.1 & 2.5 & 2.3 &
6.9 & 1.7 & 1.0 & 1.4 & - \\
$\textrm{JES}~\nvtx$ & - & 2.3 & 2.7 & 0.5 & 0.7 & 0.2 & 6.3 & - \\
$\textrm{JES AFII}$ & 0.1 & 0.5 & - & 3.4 & 1.4 & 0.3 & - & - \\
$\textrm{JER}$ & 0.7 & 1.4 & 0.3 & 3.5 & 3.1 & 3.1 & 4.1 & - \\
$b\textrm{-tag SF}$ & - & 2.2 & 5.7 & - & - & - & 0.1 & - \\
$\textrm{light tag SF}$ & 1.3 & 1.5 & 1.3 & 1.4 & 1.6 & 1.5 & 2.4 & -
\\
$c\textrm{-tag SF}$ & 0.6 & 0.2 & - & 1.0 & 0.5 & - & - & - \\
$\calomet~\textrm{scale}$ & - & - & - & - & - & - & - & - \\
$\calomet~\textrm{reso.}$ & - & - & - & - & - & - & - & - \\
$\trkmet~\textrm{scale}$ & 0.3 & 2.4 & 4.7 & 0.1 & 1.0 & 0.6 & 2.5 & -
\\
$\trkmet~\textrm{reso.}$ & 0.2 & 2.9 & 6.1 & 1.4 & 1.0 & 6.1 & 2.1 & -
\\
$\textrm{di-jet fake rate}$ & - & 0.5 & - & - & - & - & - & 12.8 \\
$\textrm{fake rate}~\mu$ & - & 0.3 & - & - & - & - & - & 7.9 \\
$\textrm{fake rate}~e$ & - & 0.9 & - & - & - & - & - & 24.4 \\
$\left \langle \mu \right \rangle~\textrm{re-scale}$ & - & - & - & - &
- & - & - & - \\
\hline
\end{tabular}
}
\caption[Instrumental uncertainty summary in the \emme
  channel for the $7 \tev$ analysis.]{Breakdown of instrumental
  uncertainties in the \emme
  channel for the $7 \tev$ analysis. For predictions which are purely from MC simulation,
  uncertainties are
  the variation in the event yield in the BDT SR. For top processes,
  the uncertainty is the variation in $\alpha$.}
\label{chap:analysis:tab:exp_sys_df}
\end{center}
\end{table}

\begin{table}[p!]
\begin{center}
\renewcommand{\arraystretch}{1.2}
\resizebox{0.9\textwidth}{!}{
\begin{tabular}{l || c c | c c c c c c }
\hline
Uncertainty & \multirow{2}{*}{Signal} & Total &
\multirow{2}{*}{top} & \multirow{2}{*}{$WW$} & \multirow{2}{*}{ggF} &
Non-$WW$ & \multirow{2}{*}{\ZDY} & \multirow{2}{*}{Fakes} \\
Source & & Back & & & & Diboson & & \\
\hline
$\textrm{Trigger}$ & 0.5 & 0.2 & 0.1 & - & 0.5 & 0.2 & 0.2 & - \\
$\textrm{electron SF}$ & 1.1 & 0.4 & 0.8 & 1.0 & 1.3 & 2.0 & - & - \\
$\textrm{electron reso.}$ & - & 0.1 & - & 0.9 & 0.8 & 0.6 & 0.1 & - \\
$\textrm{electron scale}$ & - & 1.0 & 4.4 & 0.2 & 0.4 & 0.6 & 0.5 & -
\\
$\textrm{electron iso.}$ & 1.5 & 0.9 & 1.3 & 1.9 & 1.7 & 3.3 & 0.3 & -
\\
$\textrm{muon SF}$ & - & 0.1 & - & - & - & - & 0.1 & - \\
$\textrm{muon reso.}$ & - & - & - & 0.1 & - & 0.1 & - & - \\
$\textrm{muon scale}$ & - & - & - & - & - & - & - & - \\
$\textrm{muon iso.}$ & 1.3 & 0.7 & 1.4 & 1.1 & 1.2 & 0.4 & 0.4 & - \\
$\textrm{JES detector}$ & 1.3 & 2.1 & 9.5 & 0.7 & 2.8 & - & 0.3 & - \\
$\textrm{JES}~\eta~\textrm{stat.}$ & 0.7 & 1.2 & 3.8 & 1.8 & 1.8 & - &
0.3 & - \\
$\textrm{JES modelling}$ & 1.1 & 2.0 & 5.1 & 3.3 & 4.0 & 1.1 & 0.7 & -
\\
$\textrm{JES Stat.}$ & 0.3 & 1.1 & 3.8 & 1.7 & 0.7 & - & 0.3 & - \\
$\textrm{JES}~b~\textrm{jet}$ & - & 1.5 & 8.2 & 0.1 & - & - & - & - \\
$\textrm{JES}~\eta~\textrm{model}$ & 5.5 & 4.8 & 8.9 & 13.5 & 9.2 &
6.8 & 0.4 & - \\
$\textrm{JES flavor}$ & 3.5 & 3.1 & - & 0.2 & 6.8 & 0.2 & 5.3 & - \\
$\textrm{JES flav. response}$ & 2.1 & 1.5 & 1.3 & 6.1 & 5.1 & 1.1 & -
& - \\
$\textrm{JES}~\left \langle \mu \right \rangle$ & 1.5 & 1.8 & 8.1 &
4.0 & 2.8 & 0.6 & 1.0 & - \\
$\textrm{JES}~\nvtx$ & 0.1 & 1.8 & 8.2 & 0.7 & - & - & 0.4 & - \\
$\textrm{JES AFII}$ & 0.4 & 1.7 & 4.8 & 1.4 & 0.1 & 0.2 & 1.3 & - \\
$\textrm{JER}$ & 0.3 & 4.8 & 16.3 & 2.3 & 2.4 & 1.2 & 4.3 & - \\
$b\textrm{-tag SF}$ & - & 0.9 & 5.1 & - & - & - & - & - \\
$\textrm{light tag SF}$ & 1.3 & 1.0 & 1.3 & 1.7 & 1.6 & 1.8 & 0.6 & -
\\
$c\textrm{-tag SF}$ & 0.6 & - & - & 0.1 & - & - & - & - \\
$\calomet~\textrm{scale}$ & - & 0.8 & 4.4 & - & - & - & - & - \\
$\calomet~\textrm{reso.}$ & - & - & - & 0.2 & 0.1 & - & - & - \\
$\trkmet~\textrm{scale}$ & - & 1.3 & 3.6 & 0.7 & 1.4 & 0.1 & 1.3 & -
\\
$\trkmet~\textrm{reso.}$ & 0.3 & 1.5 & 6.3 & 2.6 & 1.5 & 6.2 & 0.5 & -
\\
$\textrm{di-jet fake rate}$ & - & - & - & - & - & - & - & 0.2 \\
$\textrm{fake rate}~\mu$ & - & 2.2 & - & - & - & - & - & 37.9 \\
$\textrm{fake rate}~e$ & - & 0.3 & - & - & - & - & - & 4.8 \\
$\left \langle \mu \right \rangle~\textrm{re-scale}$ & - & - & - & - &
- & - & - & - \\
\hline
\end{tabular}
}
\caption[Instrumental uncertainty summary in the \eemm
  channel for the $7 \tev$ analysis.]{Breakdown of instrumental
  uncertainties in the \eemm
  channel for the $7 \tev$ analysis. For predictions which are purely
  from MC simulation,
  uncertainties are
  the variation in the event yield in the BDT SR. For top processes,
  the uncertainty is the variation in $\alpha$.}
\label{chap:analysis:tab:exp_sys_sf_2011}
\end{center}
\end{table}


\subsection{Results}
\label{chap:analysis:sec:reanalysis:subsec:results}

The predicted and observed event yields for the $7 \tev$ analysis are
shown in table~\ref{chap:analysis:tab:signal_region_cutflow_2011}. In
the \emme channel, the expected number of background events in the
first bin is 3.5, with 6 events observed, and in the second BDT bin
0 events are observed with the expected background prediction of 0.9
events. In the single \eemm BDT bin, 4.7 background events are
predicted with 0.8 signal events, and the number of events observed is
3. 

\begin{table}[h]
\begin{center}
\renewcommand{\arraystretch}{1.2}
\resizebox{1.0\textwidth}{!}{
%\begin{tabular}{ +l^| c^ c^ c^ | c^ c^ c^ c^ c^ c }
\begin{tabular}{l || c c c | c c c c c c }
\hline
% & & & Total & & & & Non-$WW$ & & \\
\multirow{2}{*}{Cut stage} & \multirow{2}{*}{Observed} &
\multirow{2}{*}{Signal} & Total &
\multirow{2}{*}{top} & \multirow{2}{*}{$WW$} & \multirow{2}{*}{ggF} &
Non-$WW$ & \multirow{2}{*}{\ZDY} & \multirow{2}{*}{Fakes} \\
 & & & Back & & & & Diboson & & \\
\hline
$\Njet \geq 2$ & $155205$ & $16$ & $151286$ & $13809$ & $289$ & $27$ &
$226$ & $135668$ & $1268$ \\
\hline
$\boldsymbol{\emme~\textrm{\bf channel}}$ & $8042$ & $8$ & $7668$ & $6831$ & $143$ &
$13$ & $54$ & $507$ & $119$ \\
$\Nbjet = 0$ & $949$ & $5$ & $950$ & $372$ & $102$ & $10$ & $39$ &
$363$ & $65$ \\
$\textrm{CJV}$ & $799$ & $5$ & $798$ & $294$ & $91$ & $8$ & $34$ &
$310$ & $61$ \\
$\textrm{OLV}$ & $194$ & $3$ & $172$ & $66$ & $18$ & $2$ & $6$ & $65$
& $14$ \\
$Z\rightarrow{\tau\tau}~\textrm{veto}$ & $100$ & $2$ & $92$ & $43$ &
$11$ & $2$ & $4$ & $21$ & $10$ \\
$\textrm{BDT} > -0.48$ & $6$ & $1$ & $5$ & $2$ & $1$ & $1$ & $0$ & $1$
& $0$ \\
\hline
{\bf\color{blue} BDT bin 1} & $6$ & $0.6 \pm 0.0$ & $3.5 \pm 0.5$ & $1.1$ &
$0.6$ & $0.4$ & $0.4$ & $0.9$ & $0.1$ \\
{\bf\color{blue} BDT bin 2} & $0$ & $0.9 \pm 0.0$ & $0.9 \pm 0.2$ & $0.4$ &
$0.3$ & $0.2$ & $0.0$ & $0.0$ & $0.0$ \\
\hline
$\boldsymbol{\eemm~\textrm{\bf channel}}$ & $147163$ & $8$ & $143618$ & $6978$ & $146$
& $14$ & $172$ & $135160$ & $1149$ \\
$\calomet > 45 \gev$ & $10251$ & $5$ & $9560$ & $5054$ & $103$ & $8$ &
$85$ & $4230$ & $79$ \\
$\trkmet > 40 \gev$ & $6415$ & $4$ & $6108$ & $4691$ & $96$ & $7$ &
$75$ & $1200$ & $39$ \\
$Z~\textrm{veto}$ & $4272$ & $4$ & $4134$ & $3697$ & $76$ & $7$ & $22$
& $310$ & $22$ \\
$\mll < 75 \gev$ & $2322$ & $4$ & $2221$ & $1888$ & $37$ & $7$ & $13$
& $258$ & $18$ \\
$\Nbjet = 0$ & $328$ & $3$ & $322$ & $106$ & $26$ & $5$ & $9$ & $173$
& $4$ \\
$\textrm{CJV}$ & $261$ & $3$ & $263$ & $83$ & $23$ & $4$ & $8$ & $142$
& $4$ \\
$\textrm{OLV}$ & $56$ & $2$ & $56$ & $20$ & $5$ & $1$ & $2$ & $27$ &
$1$ \\
$Z\rightarrow{\tau\tau}~\textrm{veto}$ & $39$ & $1$ & $36$ & $17$ &
$4$ & $1$ & $1$ & $12$ & $1$ \\
\hline
{\bf\color{blue} BDT bin 1} & $3$ & $0.8 \pm 0.0$ & $4.7 \pm 1.5$ & $0.7$ &
$0.6$ & $0.2$ & $0.2$ & $2.8$ & $0.2$ \\
\hline
\end{tabular}
}
\caption[]{Observed and expected event yields for the $7 \tev$
analysis at each cut stage,
starting with the \Njet cut. Expected event yields are split into the
background components. The expected signal includes VBF and VH
contributions. Highlighted in {\color{blue}blue} are the yields in the
three signal region BDT bins, which go into the likelihood fit. The uncertainties
shown are statistical only. Normalization factors are applied to top
and \ZDY in the SR, but not at other cut stages.}
\label{chap:analysis:tab:signal_region_cutflow_2011}
\end{center}
\end{table}



