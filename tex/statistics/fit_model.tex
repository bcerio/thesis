In the VBF BDT analysis, a likelihood of the form shown in
equation~\ref{chapter:statistics:equation:LH_bin_full} is defined. The
term that represents the measurement in the signal region is a Poisson
product over the lepton flavor channels and the three BDT bins:

\begin{equation}
\mathscr{L}_{\textrm{SR}}(\mu,\vec{\mu}^{\prime},\vec{\theta}) =
\prod_{\textrm{bin }i}^{1,2,3} \prod_{j}^{\emme,\eemm}
P[N_{ji} \,|\,
\mu\cdot s_{ji}(\vec{\theta}) + \sum_{\ell}^{\textrm{top},\ZDY}
\mu^{\ell}_{i}\cdot b^{\ell}_{ji}(\vec{\theta}) + \sum_{k}
b^{k}_{ji}(\vec{\theta})]
\label{chapter:statistics:equation:LH_bdt_SR}
\end{equation}

\noindent
Background processes indexed by $\ell$, top and \ZDY, are constrained with an auxiliary
strength measurement $\mu^{\ell}_{i}$. In the case of top, this
parameter is shared between the two channels, while for \ZDY, the
strength parameter is applied only to the \eemm SR bins. The background predictions indexed by $k$,
$b^{k}_{ji}$, are derived from MC simulation, except for backgrounds
from fake leptons, which are estimated with a data-driven technique
(section~\ref{chap:analysis:sec:dd_backgrounds:subsec:fakes}) that is
external to the likelihood. The next term,
representing the auxiliary measurement of the $\mu^{\ell}_{i}$, can be
written explicitly as 

\begin{equation}
\mathscr{L}_{\textrm{CR}}(\mu,\vec{\mu}^{\prime},\vec{\theta}) =
\prod_{k}^{\textrm{top},\ZDY} \prod_{\textrm{bin }i}^{1,2}
P(M_{i}^{k} \,|\, \mu_{i}^{k}\cdot \beta^{k}_i)
\label{chapter:statistics:equation:LH_bdt_auxiliary}
\end{equation}

\noindent
where $\beta$ is the background $k$ prediction from MC in bin $i$. The
product over bins does not run over the three SR bins, because a
common strength parameter is applied in the two high SR bins, i.e. in
equation~\ref{chapter:statistics:equation:LH_bdt_SR}, $\mu^{\ell}_{2}
= \mu^{\ell}_{3}$. The likelihood terms defined in equations~\ref{chapter:statistics:equation:LH_bdt_SR}
and~\ref{chapter:statistics:equation:LH_bdt_auxiliary} are shown
pictorally in figure~\ref{chap:analysis:fig:fit_model}. The final term
contrains the NPs based on auxiliary measurements:

\begin{equation}
\mathscr{L}_{\textrm{NP}}(\vec{\theta}) = \prod_i^{N_{\textrm{NP}}}
\mathscr{G}(\tilde{\theta}_i \,|\,
\theta_i,1) \prod_j P(\tilde{\theta}_j \,|\, \theta_j M_j)
\label{chapter:statistics:equation:LH_bdt_NP}
\end{equation}

\noindent
The first term is a product over all NPs which do not represent
statistical uncertainties, including both theory and instrumental
systematics. Each NP is constrained with a single unit Gaussian
term. The additional term, with index $j$ running over all SR and CR bins in
equations~\ref{chapter:statistics:equation:LH_bdt_SR}
and~\ref{chapter:statistics:equation:LH_bdt_auxiliary}, is the Poisson
response to account for statistical uncertainties on the MC
prediction. In each bin, all processes are combined for a single
term. 

\begin{figure}[h]
  \centering
  \includegraphics[width=0.95\textwidth]{fig/statistics/fit_model_tiff_1.eps}
   \caption{Summary of the likelihood terms,
  $\mathscr{L}_{\textrm{SR}}$ and $\mathscr{L}_{\textrm{CR}}$ for the VBF
  analysis, split into lepton channels. The top CR (c) is used to
  constrain the \ttbar and single top normalizations in both flavor
  channels. The \ZDY CR constrains \ZDYll in the \eemm channel only.}
  \label{chap:analysis:fig:fit_model}
\end{figure}

The likelihood model for the $7 \tev$ analysis is adjusted to account for
the fact that there are only two (one) BDT bins in the \emme (\eemm)
channel. In addition, the $\mathscr{L}_{\textrm{NP}}$ term is modified
for cases in which the uncertainty sources are not the same as those
in the $8 \tev$ analysis. When the uncertainties are from the same
source, they are represented by the same term in
$\mathscr{L}_{\textrm{NP}}$, and are therefore fully correlated. 

Higgs production by gluon fusion receives special treatment, as there
is information to be gained from other \hwwlnln
measurements about the rate of ggF. In particular, due to the relative dominance of ggF in
the zero and one jet bins, terms associated with the measurements in
these bins are added to the likelihood~\cite{}. Moreover, an additional
measurement in the two jet bin, optimized to measure ggF and orthogonal
to the VBF regions, is included in the likelihood. In these
additional terms, the normalization of ggF is promoted to a nuisance
parameter, and correspondingly, the ggF term in
$\mathscr{L}_{\textrm{SR}}$ is changed from $b^{\textrm{ggF}}_{ji}$ to
$\mu_{\textrm{ggF}}\cdot b^{\textrm{ggF}}_{ji}$. The auxiliary ggF
measurements bring with them another set of
$\mathscr{L}_{\textrm{NP}}$ and $\mathscr{L}_{\textrm{CR}}$ terms,
which are collectively denoted as
$\mathscr{L}_{\textrm{other}}(\mu_{VBF},\mu_{ggF},\vec{\mu}_b,\vec{\theta})$. The
auxiliary $\mu_{ggF}$ measurements are subject to the same
instrumental systematics as the VBF analysis, motivating the
correlation of all such NP terms in the likelihood. In most cases,
uncertainties from theoretical sources are not correlated, as they
were evaluated in different phase space regions. The final likelihood
can then be written as the product of the likelihoods outlined above:

\begin{equation}
\mathscr{L}_{\textrm{full}} = \mathscr{L}_{\textrm{SR}}\cdot\mathscr{L}_{\textrm{CR}}\cdot\mathscr{L}_{\textrm{NP}}\cdot\mathscr{L}_{\textrm{2011}}\cdot\mathscr{L}_{\textrm{other}}.
\end{equation}

