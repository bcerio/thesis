
The BDT algorithm, described in section XX, is well-suited for the
classification of VBF signal and its associated backgrounds, as the
final state includes many correlated physics objects. The BDT training
algorithm builds a model $F(\vect{x})$ that maps the input variables
\vect{x} to a number $y$ in the range [-1,1], with -1 corresponding to
events that are background-like and 1 corresponding to those that are
signal-like. Geometrically, exploiting differing correlations among
BDT inputs in the signal and background, the BDT defines a signal-rich hypercube in
phase space, with events that fall in the regions that are most pure
in signal being assigned a $y$ value close to 1 and events falling on the
periphery being de-weighted to lower values of $y$. In the following,
the BDT response $y$ will be referred to as BDT or BDT score.

\subsection{Higgs Decay}

There are eight BDT
inputs: \dphill, \mll, \mT, \mjj, \dyjj, \pTtot, \lepEtaCent,
and \SumMlj. These can be categorized into those that are
sensitive to the decay of the Higgs into two $W$ bosons and those that
pick out the products of Higgs production via VBF. The
quantity \dphill is an example of the former. As discussed in
section XX, the final state leptons in the decay chain \hwwlnln fall close in
azimuthal angle due to the spin of the SM Higgs boson and the V-A
structure of the weak force. \dphill, which is the
absolute value of the difference in $\phi$ between the two leptons,
shown at the pre-selection stage in figure~\ref{chap:analysis:fig:bdt_inputs_mconly_df}, is peaked near zero
for signal. This distribution is relatively flat for the dominant
backgrounds top and $WW$. Another variable that picks out the Higgs
decay topology is \mll. Because \mll~$\simeq|p_1||p_2|(1-\cos(\dphill))$
and signal peaks at $\dphill \sim 0$, the signal lies at low
values of \mll, while top, continuum $WW$ and \ZDY fall at higher
values. 

\begin{figure}[p!]
  \centering
   \includegraphics[width=0.4\textwidth]{fig/analysis/bdt_input_mconly/emme_CutZttVeto_2jetincl_DPhill_mh125_lin.eps}
   \includegraphics[width=0.4\textwidth]{fig/analysis/bdt_input_mconly/emme_CutZttVeto_2jetincl_Mll_mh125_lin.eps}
   \includegraphics[width=0.4\textwidth]{fig/analysis/bdt_input_mconly/emme_CutZttVeto_2jetincl_DYjj_mh125_lin.eps}
   \includegraphics[width=0.4\textwidth]{fig/analysis/bdt_input_mconly/emme_CutZttVeto_2jetincl_Mjj_mh125_lin.eps}
   \includegraphics[width=0.4\textwidth]{fig/analysis/bdt_input_mconly/emme_CutZttVeto_2jetincl_Pttot_tr_mh125_lin.eps}
   \includegraphics[width=0.4\textwidth]{fig/analysis/bdt_input_mconly/emme_CutZttVeto_2jetincl_MT_tr_mh125_lin.eps}
   \includegraphics[width=0.4\textwidth]{fig/analysis/bdt_input_mconly/emme_CutZttVeto_2jetincl_SumOFMvaMLepxJety_mh125_lin.eps}
   \includegraphics[width=0.4\textwidth]{fig/analysis/bdt_input_mconly/emme_CutZttVeto_2jetincl_contOLV_mh125_lin.eps}
   \caption{Distributions
   of the eight BDT inputs \dphill, \mll, \dyjj, \mjj, \pTtot, \mT, \SumMlj, and \lepEtaCent
   in the \emme channel after pre-selection. Signal is enhanced by a
   factor of 50 to illustrate the separation between signal and background.}
  \label{chap:analysis:fig:bdt_inputs_mconly_df}
\end{figure}

\begin{figure}[p!]
  \centering
   \includegraphics[width=0.4\textwidth]{fig/analysis/bdt_input_mconly/eemm_CutZttVeto_2jetincl_DPhill_mh125_lin.eps}
   \includegraphics[width=0.4\textwidth]{fig/analysis/bdt_input_mconly/eemm_CutZttVeto_2jetincl_Mll_mh125_lin.eps}
   \includegraphics[width=0.4\textwidth]{fig/analysis/bdt_input_mconly/eemm_CutZttVeto_2jetincl_DYjj_mh125_lin.eps}
   \includegraphics[width=0.4\textwidth]{fig/analysis/bdt_input_mconly/eemm_CutZttVeto_2jetincl_Mjj_mh125_lin.eps}
   \includegraphics[width=0.4\textwidth]{fig/analysis/bdt_input_mconly/eemm_CutZttVeto_2jetincl_Pttot_tr_mh125_lin.eps}
   \includegraphics[width=0.4\textwidth]{fig/analysis/bdt_input_mconly/eemm_CutZttVeto_2jetincl_MT_tr_mh125_lin.eps}
   \includegraphics[width=0.4\textwidth]{fig/analysis/bdt_input_mconly/eemm_CutZttVeto_2jetincl_SumOFMvaMLepxJety_mh125_lin.eps}
   \includegraphics[width=0.4\textwidth]{fig/analysis/bdt_input_mconly/eemm_CutZttVeto_2jetincl_contOLV_mh125_lin.eps}
   \caption{Distributions of the eight BDT inputs
   \dphill, \mll, \dyjj, \mjj, \pTtot, \mT, \SumMlj, and \lepEtaCent
   in the \eemm channel after pre-selection. Signal is enhanced by a
   factor of 50 to illustrate the separation between signal and background.}
  \label{chap:analysis:fig:bdt_inputs_mconly_sf}
\end{figure}

The transverse mass of the \wwlnln system
(\mT)~\cite{bib:PhysRevD:43779,bib:Barr:2009mx} is
designed to capture the mass of the Higgs and is therefore a powerful
discriminant against non-resonant $WW$ background. With neutrinos in
the final state, it is not possible to fully reconstruct the mass of
the Higgs, $m_H = \sqrt{(p_{\ell 1} + p_{\ell 2} + p_{\nu 1} + p_{\nu
2})^2}$, because the neutrino 4-vectors are not reconstructed. Instead,
the invariant mass in the transverse direction is used. This quantity
can be expressed as the sum of two 4-vectors: $m_T^2 = (p_{\ell\ell} +
p_{\nu\nu})^2$. Expanding the sum yields

\begin{equation}
\label{chap:analysis:equation:mT_1}
m_T^2 = m_{\ell\ell}^2 + m_{\nu\nu}^2 +
2(E_{T}^{\ell\ell}E_{T}^{\nu\nu}-\vect{p}_T^{\,\ell\ell}\cdot\vect{p}_T^{\,\nu\nu}).
\end{equation}

\noindent In the above equation, the transverse energy $E_{T}^{\nu\nu}$
is equivalent to \etmiss. Setting $m_{\nu\nu}^2 = E_{T}^{\nu\nu\,2} -
|\vect{p}_T^{\nu\nu}|^2$, the definition in
equation~\ref{chap:analysis:equation:mT_1} can be re-expressed as

\begin{equation}
\begin{aligned}
\label{chap:analysis:equation:mT_2}
m_T^2 & = E_{T}^{\ell\ell\,2} - |\vect{p}_T^{\ell\ell}|^2 +
E_{T}^{\nu\nu\,2} - |\vect{p}_T^{\nu\nu}|^2 +
2E_{T}^{\ell\ell}E_{T}^{\nu\nu} -
2\vect{p}_T^{\,\ell\ell}\cdot\vect{p}_T^{\,\nu\nu} \\
& = (E_{T}^{\ell\ell} + \etmiss)^2 - (\vect{p}_T^{\,\ell\ell} +
\vect{E}_{\mathrm{T}}^{\rm miss})^2
\end{aligned}
\end{equation}

\noindent
where $E_{T}^{\ell\ell}$ and $\vect{p}_T^{\,\ell\ell}$ are the transverse
energy and transverse momentum of the dilepton system,
and $\vect{E}_{\mathrm{T}}^{\rm miss}$ is the track \etmiss vector. As defined
above, if $H$ is at rest, \mT is bounded from above by $m_H$. This
edge in the distribution is smeared out because the Higgs boson is not
always produced at rest. Moreover, the slope of the edge is lessened due to
detector resolution effects and the finite, although small, decay
width of $H$. The shape of the distribution for signal and background
is shown in figure~\ref{chap:analysis:fig:bdt_inputs_mconly_df}. Due to its discriminating power against
continuum $WW$ and its dependence on $m_H$, the \mT distribution is
used in the binned likelihood fit for the 0,1j bins and the cut-based VBF
analysis, instead of applying a cut on this distribution. 

\subsection{VBF Topology}

The remaining five BDT inputs isolate the region of phase space associated with
the VBF topology. The defining characteristic of VBF is two ``tag''
jets with a large rapidity gap. The rapidity distance between the two
leading jets, \dyjj, is therefore used as a BDT input. As shown in
figure~\ref{chap:analysis:fig:bdt_inputs_mconly_df}, signal peaks at \dyjj~$\sim 4.5$, with
the \ttbar~peaking at \dyjj~$\sim 2$. The dijet invariant mass, \mjj, is also
a BDT input. This quantity is a function of \dyjj:

\begin{equation}
%\renewcommand{\arraystretch}{1.2}
\begin{aligned}
\label{chap:analysis:equation:mjj}
\mjj^2 & = m_{j1}^2 + m_{j2}^2 + 2\left [E_{\,\mathrm{T}}^{\,j1}E_{\,\mathrm{T}}^{\,j2}\cosh{(\dyjj)}
- \vect{p}_{\,\mathrm{T}}^{\,j1}\cdot\vect{p}_{\,\mathrm{T}}^{\,j1}\right ] \\
& \simeq 2p_{\,\mathrm{T}}^{\,j1}p_{\,\mathrm{T}}^{\,j2}\left
[\cosh{(\dyjj)}-\cos{(\Delta{\phi}_{jj})}\right ].
\end{aligned}
\end{equation}

\noindent
The second equality assumes that the mass of each jet is small with
respect to the jet \pt. Given that \mjj grows as $\cosh{(\dyjj)}$, and
signal lies at high \dyjj, VBF also lies at high \mjj. In spite of the
high degree of correlation between \dyjj and \mjj, \mjj contributes
discriminating power through the other terms in
equation~\ref{chap:analysis:equation:mjj}, namely the coefficient
$p_{\,\mathrm{T}}^{\,j1}p_{\,\mathrm{T}}^{\,j2}$ and the
$\cos{(\Delta{\phi}_{jj})}$ term. Jets from \ttbar~tend to be
back-to-back, while VBF jets are fairly uniformly distributed in
$\phi$. Therefore, on average, at a fixed \mjj, \dyjj is smaller
for \ttbar. Moreover, the BJV favors \ttbar~events with an ISR jet and
a b-jet outside of the tracking volume, resulting in a softer
jet \pt~spectrum for \ttbar. The coefficient is therefore smaller on
average, shifting the \mjj distribution to lower values. 

Another BDT input that enhances the VBF signal is \pTtot, the modulus
of the vector sum of all of the physics objects in the event:

\begin{equation}
\begin{aligned}
\label{chap:analysis:equation:pttot}
\pTtot = |\vpT^{l1}{+}\vpT^{\rm l2}{+}\vmet{+}\sum\vpT^{\rm
jets}|.
\end{aligned}
\end{equation}

\noindent
The sum over jets runs over all jets that pass the jet selection
criteria described in section~\ref{chap:analysis:sec:objects}. 
Track \vmet is used in $\pTtot$, and because this version of \etmiss
is merely the
vector sum of all of the physics objects in the event with an
additional soft track term, there is cancellation in the above
expression, resulting in

\begin{equation}
\begin{aligned}
\label{chap:analysis:equation:pttot_2}
\pTtot = |\vect{E}_{\mathrm{T,soft}}^{\mathrm{miss,trk}}|.
\end{aligned}
\end{equation}

\noindent
This quantity is the modulus of the vector sum of soft tracks
falling outside of selected jets, which probes the amount of soft QCD
radiation in a given event. Backgrounds with QCD jets have more soft
gluon radiation and lie at larger values of \pTtot with respect to VBF
events which have relatively little QCD activity. 

Lepton eta centrality (\lepEtaCent) is an extension of the OLV
discussed in section~\ref{subsec:vbf_select}. It is a measure of the
centrality of the leptons with respect to the two tag jets, and is
defined by the following equations:

\begin{eqnarray}
&& \eta_{l0} \, \textrm{cent.} = 2 \cdot
|\frac{\eta_{l_0}-\bar{\eta}}{\eta_{j_0}-\eta_{j_1}}|  \nonumber\\
&& \eta_{l1} \, \textrm{cent.} = 2 \cdot
|\frac{\eta_{l_1}-\bar{\eta}}{\eta_{j_0}-\eta_{j_1}}|  \nonumber\\
&&\nonumber \\
&& \eta_{\mathrm{lep}} \, \textrm{centrality}
= \eta_{l0} \, \textrm{cent.} + \eta_{l1} \, \textrm{cent.}.
\label{eqn:contOLV_def}
\end{eqnarray}

\noindent
where $\bar{\eta} = (\eta_{j_0}+\eta_{j_1})/2$. As defined above, for
a given lepton, the quantity $\eta_{l} \, \textrm{cent.}$ is zero if
the the lepton falls directly in between the tag jets, is less than
one if the lepton falls at some $\eta$ value between the tag jets, and is greater than one if
it falls outside of the tag jets. The OLV, which requires that one
lepton falls outside of the pseudorapidity gap, is equivalent to
applying the cuts $\eta_{l0} \, \textrm{cent.} < 1$ and 
$\eta_{l1} \, \textrm{cent.} < 1$. After pre-selection, \lepEtaCent is therefore constrained to the range [0,2]. Because
leptons in top processes fall closer to the $b$-jets, the top background
events that do survive the OLV have leptons with $\eta_{l} \, \textrm{cent.}$ closer to
one. The more central VBF leptons fall closer to $\bar{\eta}$
and \lepEtaCent is smaller, as illustrated in figure~\ref{chap:analysis:fig:bdt_inputs_mconly_df}.

The final BDT input is the sum of the lepton-jet invariant masses for
all lepton-jet pairs, denoted \SumMlj. The sum runs over the two tag (leading)
jets, and not the other selected jets. Like \lepEtaCent, this
observable is designed to exploit the fact that VBF jets fall at high
pseudorapidity, while the Higgs decay leptons are central. This large
opening angle between the leptons and jets results in large $M_{\ell
j}$ terms compared to background. 

The BDT input distributions are shown for signal and background for
the \emme (\eemm) channel in figure~\ref{chap:analysis:fig:bdt_inputs_mconly_df} (\ref{chap:analysis:fig:bdt_inputs_mconly_sf}). 

\subsection{Input Performance}

Do I want to show the table of the discriminating power of the BDT inputs?


