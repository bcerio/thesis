
Interesting scattering processes nearly always include charged
particles in the final state. The two tracking systematics in ATLAS, the ID and the muon
system, are designed to measure the trajectories of these charged
particles, from which momentum 4-vectors are derived. A track is fully
specified by five parameters $\vect{\alpha} = ()$. In the following section,
the algorithms for fitting these parameters from detector hits will be
described. 

\subsection{Inside-out Tracks}

The primary tracking algorithm in ATLAS builds tracks starting with
hits in the pixel detectors, and progressively adding hits at larger
$r$ values. This is known as the inside-out track reconstruction
paradigm~\cite{bib:Cornelissen:2007vba}. Starting with hits from the pixel and SCT detectors,
collectively known as the silicon detectors, clusters of contiguous
hits are identified and associated with a coordinate in space. The
track-finding algorithm is seeded by a group of three clusters in
different layers that is consistent with a track. This seed defines a
``road'' along which the track candidate is built. The track
trajectory is propagated through the detector in a Kalman filter based
approach~\cite{bib:Fruhwirth:1987fm}. At the $k^{\textrm{th}}$ detector layer, the hit
that is most consistent with the track parameters from the previous
layer is assigned to the track. The track parameter vector is then
updated with the new hit included, yielding $\vect{\alpha}_k$. The propagation to the subsequent
layer is then given by

\begin{equation}
\vect{\alpha}_{k+1} = M_k\vect{\alpha}_{k} + \vect{\epsilon}_{k}
\end{equation}

\noindent
where $M_k$ is the linear map representing the magnetic field between the
two detector layers, and $\epsilon_{k}$ is a stochastic term that
accounts for multiple scattering of the charged particle. The
procedure is then repeated at layer $k+1$. The above approach for
fitting the track parameter vector is equivalent to a global minimum
least-squares fit. 

After all of the seeds have been evaluated by the Kalman-filter
algorithm, ambiguities are removed from the resulting track candidate
collection. These ambiguities include track candidates that share one
or more hits, are incomplete, or include hits that arise from more
than one charged particle. In order to resolve these ambiguities, the
tracks are first refitted using a detector geometry with a more
realistic description of the detector material. Each track is then
assigned a score that quantifies the quality of the track. Tracks with
more hits receive higher scores, and these scores are weighted
according to the precision of the detector subsystem in which the hit
reside. Given the high intrinsic efficiencies of the silicon
detectors, if a hit is absent along a track trajectory, the track
score is penalized. After track scoring, hits that are shared by more
than one track are assigned to the track with higher score. The
remaining track is then refitted and re-scored. Tracks that fail to
pass a score threshold are discarded.

The silicon only track candidates that pass the quality threshold are
the seeds for the extension of the track into the TRT. In
the first pass, the silicon hits that define the track seed for the
TRT extension are not changed. Again, the Kalman filter approach is
used to map the track from one detector surface to another. Along the
trajectory, TRT hits are added depending on the distance between the
Kalman filter prediction and the actual hit location. With the TRT hits
identified, the track is refitted with the hits from all
subdetectors-- silicon and TRT. In this pass, the silicon hits are
allowed to vary. The score of the refitted track is compared to that of
the silicon only track, and if it is smaller, the silicon only track
is retained, with the hits from the TRT as outliers, but still
associated to the track. If the score of the refitted track is larger,
this new track is retained. 

\subsection{Outside-in Tracks}

Because the inside-out algorithm is seeded by silicon clusters
associated to a primary vertex, it fails to find tracks arising from
secondary decays that occur late in the silicon system or in the
TRT. These tracks include those associated with electrons
from photon conversions and charged hadrons or leptons from hadron
decays. To recover secondary tracks, an outside-in algorithm, starting
in the TRT, is used in concert with the inside-out algorithm. 

Since TRT hits do not have information about the global $z$
coordinate in the barrel, it is not possible to use pattern-finding
procedure based on three-dimensional space coordinates as is done in
inside-out tracking. Instead, the TRT hits are projected to the
$r-\phi$ plane in which they form straight lines. The hits are then Hough
transformed~\cite{bib:Duda:1972uht} to the parameter
space of a line, where the coordinate are the $y$-intercept, or the
initial $\phi$ of the line, and the slope, or $1/p_{\textrm{track}}$,
where $p_{\textrm{track}}$ is the momentum of the track. In this
space, track candidates appear as maxima at their respective
($\phi_0,1/p_{\textrm{track}}$) coordinates. This technique, which
defines the hit $r-\phi$ coordinate to be the center of the straw,
fails to incorporate information about the drift time. In a second
step, the track candidates are refitted with a Kalman-filter-based
algorithm that incorporates information about the drift time, thereby
improving the accuracy of the fitted track parameters. 

The final step of the outside-in track reconstruction algorithm is the
extension of the TRT segments into the silicon tracker (need good
reference for this). Scanning longitudinally in a small $r-\phi$ wedge
defined by the TRT segment, the algorithm searches for at least two
clusters of SCT hits in the three outmost layers of the
SCT. Once found, the curvature defined by these two three-dimensional
points and the first TRT hit in the TRT segment is computed, and if
the result is unreasonable, the two clusters are disregarded. The SCT
clusters that are retained seed the track-fitting algorithm,
propagating the track to smaller $r$ and extracting the fitted track
parameters. 

\subsection{Vertex Reconstruction}

